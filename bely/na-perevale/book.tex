\documentclass[12pt,a4paper,oneside]{book}
\usepackage{fontspec}
\setmainfont[Mapping=tex-text]{Linux Libertine O}
\setmonofont[Scale=0.8,Mapping=tex-text]{Linux Libertine O}

\usepackage{polyglossia}
\setdefaultlanguage[spelling=modern]{russian}
\setotherlanguage{english}

\usepackage{indentfirst}
\usepackage{fancyvrb}
\fvset{xleftmargin=4cm}
\fvset{commandchars=\\\{\}}

\setcounter{tocdepth}{0}
\setcounter{secnumdepth}{1}

\begin{document}
\title{На перевале}
\author{Андрей Белый}
\date{}
\maketitle

\tableofcontents

\part{Кризис жизни}

\subsection{Вместо предисловия}
Предлагаемый „дневник” мыслей есть часть дневника, который пришлось мне вести в Швейцарии в 1915-ом и 1916 году; части из этого дневника в своё время были мной напечатаны в отрывках; другие же части вошли в мою книгу „Кризис сознания”, увы, не могущую появиться на свет по условиям нашего времени. Перечитывая этот дневник, убеждаюсь невольно: не устарел он; охвачены тем же мы легкомыслием; события, ударявшие нас, озлобляли нас друг против друга; на себя самих не повернулись доселе мы.

„О человек, познай себя!”

\begin{flushright}Андрей Белый.\end{flushright}

Москва. 1918 года, июль.\marginpar{5}

\section*{1.}

„Гремящая тишина”!

Девятнадцатый месяц со мною она в мертвом шелесте городов, в мертвом беге часов; утром, ночью и днём — все гремит с горизонта.

Есть особая тишина у швейцарской провинции этого угла Базельланда, в котором засел я давно; неподобна она тишине русских ширей, где сердце бунтует, где все — необъятно, где ветром несутся пространства; и падает небо на вас самоцветною звёздочкой; всё под ним отступило: всё — плоско; всё — ровно; глаза упираются в переливы заката и в кудри косматого облака; и тоска или радость, от который нет выхода, угоняют вас прямо в смерть.

Здесь, в Базельланде, всё — скучно; всё — скученно: несуетливо, но — душно; и внятно гласящее небо здесь часто закрыто, и внятно светящее слово\marginpar{7} зажато в гортани неповоротливых обитателей двух деревушек, между которыми поселился я до войны; жители Арнсгейма и Дорнаха не внимают давно уже голосу внятно гласящих орудий с эльзасской границы.

Битвы в Эльзасе обычны: как падение местного водопадика, эти битвы сопутствуют вашей жизни; вы их слышите: говором пушек — оттуда, с границы: вот „оно” — загремело: гремит.

И гремело так год назад; через год — отгремит ли\footnote{С той поры, как написаны эти строчки, прошло уже более двух лет; уже около двух лет я в России; а вокруг — всё „гремит”. \emph{Прим. автора}}?

Обыватели местных посёлочков собираются посмотреть на фонарь, только что поставленный меж двумя деревушками, как ходили когда-то под праздник они любоваться с холма на чуть видные огонёчки шрапнелей — \emph{оттуда}.

\section*{2.}

Нас обстал кризис жизни: на перевале сознания подстерегают нас кризисы жизни; приложенья к техническим производствам культуры плотнят нашу мысль: не живая, она превратилась в абстракцию; материальное тело абстракции — машина.\marginpar{8}

Машина восстала на нас: мир стал — мир материально-машинный: и чёрствый, и чувственный; чёрствая чувственность — роковой наш удел.

Мир природы — преставился: ненормальная вытяжка из природы его заменила; утерялась в нас „вещность“, сменяясь экстрактом; и нет нам предметов, а есть предмет-\emph{ины}.

Чревоугодие материальной культуры — продукт очерствления.

Слепота — тончит ухо, а глухота — тончит глаз: неужели же для утончения зрения мы долждны протыкать барабанную перепонку? Это было б безумием. Но на безумии этом построен рост знаний; богатство машинного мира разростаются в мире ценой оглушения, иль ценой ослепления; глухие, слепые, немые вершат нашу участь.

\emph{До столького} дожили мы! До чего доживём, я не знаю.

Мы не видели удалённых молний грозы; мы увидели зарева сожигаемых зданий; расслышали — пушки; лёгкий говор сознания и голоса сознающих ещё — всё ещё! — не расшибли, на нас глухоты; не расшибут они — в будущем.

Голоса наростающих громов культуры — гремели столетия…

Если б нам уши!\marginpar{9}

\section*{3.}

Лучшие традиции Возрождения мы столетия низводили на нет: убивали столетья конкретную значимость жизни; и — говор явлений; Возрождение призывало нас явственно: полюбить все явления мира; и в Возрождении по отношенью к явлениям жизни художник с учёным сливается; художник глядит на явление — мудро; учёный явление греет, ласкающим опытом.

Таков Леонардо: наука его красотою пронизана вся; а искусство в нём — мудро; любовные опыты — опыты Леонардо-да-Винчи.

Но опыт Роджера Бэкона из средних веков — уже пытка явленья: убийство явленья; терзанье, кромсанье его; раскромсанье предметов, убийство предметов перенесли мы в XVI век вопреки всем вершинным традициям гуманизма; часто опыт природы был пыткой ея; так: XVI век, зацветя инквизицией и закруглившись в барокко (в развратно-утончённом Style jesuite), перенёс инквизиционные приёмы терзанья, пытанья в мир целой, цветущей природы; в опытах разрушались предметы для добывания всевозможных гастрических лакомств: и вещь и реальность, как цельное нечто, распались от этого: на абстрацию (пресловутую „вещь в себе”) и\marginpar{10} на труп от конкретной реальности, на феномен, на „вещь для нас”: на продукт потребления буржуазной культуры; материальное тело культуры её превратило в… часть брюха: в отложение жировых желез, в субъективистическую отрыжку действительности; развитие философии сосредоточилось на методологической разработке всевозможных \emph{отрыжек}; и пошли рости „научные” феноменализмы и скептицизмы.

Style jesuite, развитие материальной культуры, номинализм новейшей формации философии коренятся в едином источнике: в разложении конкретного мира на абстракцию и на вытяжку для гастрических потреблений; но в гастрическом потреблении — ещё полной реальности нет, и вкусовая отрыжка комфорта — не действительность вовсе; точно так же: в теоретических выводах специальных отраслей знания перед нами не мир, а разве что… проэкционный пунктирик: да, понятия именно в наших точнейших науках сведены часто к графике; и объяснить, понять — это значит: изобразить сеть кривых и условно исчислить их; дифференцировать — еще не значить: учить пониманию; и чертить графы — не значит осмысливать.

Так первичная конкретность идеи о конкретном предмете подменяется в нас эмблемой.\marginpar{11} Эмблемами мы исчислили необходимость войны; эмблематически прикинули военные партии всего мира размеры добычи; проэкционным пунктириком изобразили учёные инженеры возможные орудия истребления; возникали науки об уничтожении себе подобных; не забуду я никогда: еще будучи гимназистом, я нашёл на столе у отца два почтеннейших кирпича, испещрённых внутри крючковатыми знаками интегралов и функций; это было два руководства; одно называлось: „О внешней баллистике” (о движении ядра вне пушечного жерла); другое же называлось: „О баллистике внутренней”. Две почтенных науки об уничтожении себе подобных блистательно развивались; и бескорыстное открытие Лейбница (дифференциальное исчисление) применили таки мы к войне; преподавание метода убивать своих ближних разработали математики, инженеры, механики, техники культурнейших, цивилизованных стран; сотни тысяч убитых убиты еще до рождения: быть убитыми предначертаны.

И знай Лейбниц, что в лучшем из миров открытие его ляжет в грядущее массовым истреблением людей, колоссальнейшей бойнею мира, — как знать: может быть, своё открытие сжёг бы он.

Мы браним нынче Круппа. Нашёлся обще\marginpar{12}ственный деятель, соединивший с Круппом… и философа Канта. Но… почему Канта именно?.. Надо брать — раньше: Лейбниц — виновник теперешней бойни народов; или вернее: за Лейбницем спрятанный, тонкий гастроном культуры, вооруженный наукою, как ножом, для… мирового разбоя. Появился же этот разбойник, как прямое наследие отношенья к явлениям жизни: в тот момент, как идея в явлении угасает, явление есть предмет потребления; но явление для меня — предстоящее всякое, „ты”; и оно, это „ты” потребление.

Вивисекционные опыты с жизнью — они породили ту бойню, в которой живём: и не Лейбниц, а ранее Лейбница появившийся Бэкон, быть может, виновник характера современной войны.

Раз идея в явлении пропадает, явление — предмет потребления; и оно начинает тогда округлять нам желудок; \emph{„капиталистическое”} проявление желудочной деятельности разростается в нас; наш желудок теперь вывисает из нас толстым брюхом; и мы — брюхоногие пауки, а не люди; конкретности жизни нам — жир; идеалы живые — пунктир на бумаге, рисующий в знаках законы… баллистики; истина есть „не сущее”;  и оттого-то в „не сущее”\marginpar{13} принимаемся мы превращать вечно сущие жизни; истребляем и рвём их вокруг.

Вместо слияния с миром — господствует: пожирание мира и раздробление мира; то есть: введение мира в желудок для накопления… жировых отложений. Человек XX века — безмясый скелет, опухающий жиром; вместо \emph{знания} и \emph{сердечного} отношения к жизни у него господствует два усвоения жизни: при помощи мозга и при помощи функций желудка; первое усвоение — \emph{„крап на ничто”} (т. е. крап электронов над бездной); и при помощи этого \emph{„дифференциального крапа”} слагает он на бумаге чертёжики пушек; усвоение же второе — чревоугодие; лишь оно одно доминирует в нём; малокровная мысль, превращённая в крап электронов, становится техникой чрева, изготовляя ему искусственные, многозубые челюсти крепостей, изборождённые пушками.

\section*{4.}

Моё окошко — в долину; цветущие, белокудрые вишни весною глядят из него; вечерами восходят закаты; в него свистит ветер — всю осень, всю зиму; над вершинами низкорослых деревьев — отчетливая черепица домов; дальше — дали, бегущие в линии голубоватых холмов; в\marginpar{14} голубоватом тумане — граница; будто бы иногда распахнётся там воздух: перед ненастьем особенно; и прорежутся темные гребни Эльзаса.

Вот оттуда то и летит:

— „Ру-ру-рууу”…

Порой отзываются стёкла окон; вдруг не выдержат; и — расплачутся; звук немецкой пушки я знаю: отчетливый, надоедливый звук; а вот это невнятное „у-у-у” — вероятно, французская пушка; говорят: из Мюльгаузена и из Бэльфора пушек доносится внятно до Дорнаха.

Так говорят эти пушки — дни, месяцы: девятнадцатый месяц; здесь, в Швейцарии, пушки молчат; но молчание здесь чревато глухим, наростающим взрывом; будут взрывы повсюду; и из груди, как жерла, оторвавшись от жил, точно бомба, взорвётся кровавое, обнажённое сердце;  человек в эти дни, точно пушка: заряжен он кризисом.

Тема кризиса сплетена с возрождением. Тема гибели мира связуема с темой рождения. Не случайны поэтому голоса, нас зовущие к выси духовной: переродиться пора!

Голоса Мережковского, Ибсена, Штирнера, Ницше, Владимира Соловьёва звучали. Звучит голос Штейнера. Выявляя нам нервы культуры,\marginpar{15} гласят очень внятно они о падении великолепных обломков культуры; и — о паденье домов: домов старого строя.

Дома — под обстрелом.

И под обстрелом, быть может, вся эта тишайшая местность: в первый месяц войны, Боже мой, что тут было; появились французы в предместьи Базеля, St. Louis; понадвинулись с севера немцы; и собирались вдавить из Эльзаса в Швейцарию, к нам, передовые французские части; поразвесили объявления о возможности битвы под Базелем; ожидали мы с часу на час здесь сигнала тревоги; по первому знаку сигнала должны были мы налегке пробираться — туда, через горы: черз кряжистый Гемпен, висящий над Дорнахом. По дорогам задвигались швейцарские пехотинцы; трещал здесь и там барабан; батареи уставились по направленью к границе; в пыли забелели султаны; и — фыркали лошади; заскрипели телеги с фуражем; а сумасшедшие, исступлённые кучки кричали, что надо бежать: нейтралитет будет попран. Говорилось тогда об обстреле домов; этот дом — не опасен, а этот — опасно поставлен.

Но опасно поставлен не дом, не окрестность, не даже кантон, не страна; вся культура — опасно поставлена; вся под обстрелом она. Все\marginpar{16} кумиры культуры — в опасности; изображения Вотана, Доннера, Логе — падут; гибель старых божеств волим мы. Старый бог, бог войны (alter Gott): должен пасть!

Рушатся представления о данной действительности; рушатся переживания её; пропадает в нас строй ощущений, будто \emph{„я”} в ней находится; пропадают реальные ощущения \emph{„я”}; действительность убежала от \emph{„я”}; утекла от него; как свинцовая гиря, стремительно погружается в глубину подсознания \emph{„я”}; его целостность точится всекипящим движением мира:

\begin{Verbatim}
В какие-то кипящие колёса
Душа моя, расплавясь, протекла.
\end{Verbatim}

\section*{5.}

Должное, реальное знание — в усвоении предмета узнания; а современное знание, сосредоточившись на методе, предмет упраздняет; предметом узнанья становиться метод; и вне метода — хаось: вращение газов в желудке.

Самые органы чувств порасшатаны современною культурою; и пропитаны — алкоголем, пропитаны — никотином; восприятия органов чувств — \emph{никотинны}; в них луг пахнет дымом; в восприятиях наших природа убита давно; пошлый\marginpar{17} рёв паровоза — неотъемлемая принадлежность обычного европейского пейзажа; и — линия телеграфных столбов; а фабричная гарь — принадлежность зари; естественных восприятий в нас нет; и оттого-то нам нужны абстракции доказательств и самая материя потребления; осязание, грубейшее чувство, оно только живо в нас.

Так предметы узнания уничтожаются нами: в процессе узнания распыляются в голове электронным пунктиром и раздробляются на зубах при введении в „чрево”. Вместо конкретного мира поэтому выростает мир в нас танцующих математических знаков (дифференциалов и функций), роящихся, точно грязные мухи над миром… желудочных отбросов.

Вот — подлинный, неприкрашенный образ материальной культуры; и вот результат: потребления мира природы, её раскромсанья на части; то есть разложение её — нами, в нас, вокруг нас; мир отсутствует в нас и вне нас; мы из мира повыпали; полетели над бездной в… действительность, несоизмеримую с некогда данной от Бога.

Действительность, нам грозящая, прорезается явственно, под покровами умерщвлённой природы; вот она показалась уже, пока видная нам в аппаратах, приборах и лупах, как мир… инфу\marginpar{18}зорий; но аппараты, приборы и лупы воистину суть: наши новые органы чувств; мы испортили наши природные органы; кто-то нам подарил мир искусственных органов, прилипающий к глазам и ушам, приростающий к безорганно висящему малокровному мозгу, протягивающему во все стороны, точно спрут, свои вялые разветвления нервов: высасывать соки природы; на обнажённые нервы, лишенные кожных покровов, насели нечистые мухи, роящиеся над сознанием нашим — математическим знаком; обнажённые нервы естественно бронируем мы: сталью, железом; бронированный сталью, бесформенный, нервный, безжалостный спрут — вот искусственный человек, приготовивший нам мировую войну.

\section*{6.}

Мир бактерий в микроскопах: „Вот я выйду из труб микроскопов; и расселюсь среди вас: бактерии заживут человеками; через трубу микроскопа вы свалитесь все в микроскоп; и — заживёте бактерией”.

Воистину, на земле мы — „как будто”; где её былой лик? Где её конкретная правда? Материальная культура — не культура земли: земля — идеально, конкретна, природна, естественна.\marginpar{19}

\section*{7.}

Люблю землю я: она — горная, кристаллически чистая масса; лишь её поверхностный слой — унавоженный перегной; унавоженный, дурно пахнущий перегной, перепачканный всевозможными отбросами, в представлении большинства производителей материальной культуры — земля. Но земля есть огонь: огонь лавовых струй; и с навозом не смешана.

И земля, — это горы.

Вспоминаю скитанья в горах; мыслишь там — ни о чём; ни о чём — свисты ветра.

Но \emph{никчёмные} мысли летают огромными ритмами; мыслью рушатся горы: в душе водишь думы; идешь себе; уж не смотришь: в полузакрытых глазах метаморфозы обставшего пейзажа сотворяются заново: полуобразом, полумыслью; там линия пиков змеится орнаментом мысли, овеянной ветром; ты — ветряный: в голове твоей ветер; останови его; и — фиксируй: он тотчас же уплотняется силлогизмами; сознанием проницаешь ты ритм вне-сознательной мысли.

Где-нибудь перекусишь; и — далее.

Сознание наблюдает, описывает проростание мыслей из красочных пятен фантазии; и — проростание в эти пятна тебя обстающих громад;\marginpar{20} громады поят тебя мыслью; она — чистая, кристальная мысль, там осевшая горной породой; и здесь вставшая — философемою, как вот эта долина; ты прошёл шесть долин; шестерично переменились рельефы; шесть систем философии пробежало вершинами: Гёте, Гердер, Новалис, Шлегель, Шеллинг и Гегель — прошли пред тобою.

Ты видел во-очию их. Твоя мысль ни о чём, пробуждённая в душу павшими пиками, осозналась; и — вот она: \emph{мысль и природа — одно}.

Человечна и мягко гуманна природа; в ней нет извращений; но человеческий взор озирает её плотоядно; человек современности на неё воззрился, как кошка на птичку; птичье пение — есть; но в зажаренном птичьем мясе нет пения, в „потреблённой” природе идея убита; и — мертвой материей противостоит она нам; материализм — вне природен.

Основа природы — природа идеи; и \emph{философия тождества} Шеллинга проверяема горным ландшафтом; слушай пенье потока; записывай точно, что встанет из пенья потока в душе у тебя; правда Гётевой мысли откроется явственно: \emph{„манифестация тайных знаков”} природы в умении понимать жизнь сознания; метаморфозы идей уподобляемы метаморфозе растительных организмов; метаморфоза идеи — кон\marginpar{21}кретна, точна, наблюдаема, описуема; и описание точной фантазии мысли и есть философия.

Вот гуманнейший лейт-мотив в мировоззрениях Гёте, Гегеля, Шеллинга, Гердера; и — других; и — вещая фантазия мысли (сознание и природа — единство в природе идеи) — гуманна; корни русского самосознания в ней, в этой мысли.

Бросать камнями в эти мысли, не значит ли — откровенно идти на разрыв с нас зовущей природою, эта природа — природа сознания нашего; но подчас преставления о природе в нас подменяются представлением о „трамвайной”, о „материальной” культуре, в которой природа — машина.

\section*{8.}

Всякий знает из нас — вот такую невнятицу: вдруг покажется, что в напряжениях материальной культуры на-двое разрывается жизнь, что машины бьют в пульсы слабеющей жизни железными пульсами; схватит такая минута стремительно в суете городов; и покажется вдруг, что вот он, фешенебельный господин Манекен с угла уличной вывески рекламирует немо толпе производство Гомункула; и — рост механической жизни покажется грозным наростом; пухнет опухоль городов — пора ампутировать опухоль: материальное тело жизни — раздуто чрезмерно.\marginpar{22}

Безотчетности эти переживаем мы все; и горизонты сознания возникают пред нами; горизонты гремят своим кризисом; материальную опухоль жизни пора ампутировать: человечеству угрожает гангрена.

Помню день.

Всё червонилось, багрянело; всё — рдело птичий свист — уносил; говоры оголтелых утёсов гремели, дрожали; рыдали потоки; землевороты бежали под небо и ясногранною вереницей плотнели в тенях, гребни резали небо; серебрились осколки их; прорезались покровы природы, прорезались навстречу природе — природа сознания; и природа в природу сознания поглядела, как Сфинкс; земли не были землями: одухотворились и жили они, как природа идеи.

\section*{9.}

Материальная культура давно отошла от земли потому, что земля — идеальная; наш мир — мир искусственных аппаратов, понятий, стремлений и похотей; этот мир — не земля, а пожалуй — какая-то „X” планета, быть может созвездие Пса, не созвездия Солнца… Не дети мы Солнца.

Солнце прежнего мира (откуда мы выпали) разве что бьётся в нас; мир идеально-конкретный, т.е., небесно-земной — разве что в нашей\marginpar{23} совести, в беспредметно поющей, ритмической Аполлоновой музыке наших душ; может быть — в поэтической сказке; подлинная земля — только он; в нём — остаток конкретности.

Мы его гнали.

Мы подзывали иную действительность; мы столетия сотворяли её; мы плели себе горькую участь: заплетали нити судьбы; мы расплавили землю в \emph{„не сущие”}, электроны и атомы; диссоциация мира в картине научных понятий имеет условный, технический, вспомогательный смысл; мы осмыслили тот смысл: возвели его в перлы „идейного” творчества; перед нами восстала картина — разрыва действительности; распадается, разрывается человек — под говором пушек.

\section*{10.}

Первое впечатленье от Сфинкса:

— „Старая обезьяна, урод, эфиоп”.

Последнее впечатление:

— „Ангел”.

У подножия Сфинкса бывал очень часто я;\marginpar{24} тайна слиянности в нём Апполона и эфиопа меня поражала всегда; эта тайна вырвала его из веков; этой тайною он притянут к XX веку до нашей эры; и — после; так оба XX века пересекаются в Сфинксе: а прибои культур пеною разбиваются о него; он глядит — из под греческой маски; и из под рожи современного футуриста-художника; злободневность в нём разорвана в Вечность; и в нём Вечность сама — злоба нашего дня; в нём слиты полярности пола (он — δΣφΙγξ; он — ή). Нет ничего злободневнее кучи трухлявого камня, на которой изваяна эта старая голова.

Ряд статей одного писателя (к сожалению разменявшего свой талант) вспоминался мне в Египте: произведения Достоевского в нём сопоставлены с… египетской графикой, как плоды одинаковых переживаний и бурь; если так это, то интимное видиние Египта мы носим с собою; его тайны — на улице, где-нибудь между Литейным и Невским, а не… в булакском музее; тысячелетия прошлого не вне нас, в — в нас они с нами; и в движении нашего пальца, в улыбке, в манерах, в цилиндре, во всём строе жизни — осуществившийся синтез Египта, Халдеи, Ассирии и т.д.; откровение ассирийского духа не в кропотливейшем изучении мало понятных письмен, а в — газетной статье, может быть; эсотерика — в\marginpar{25} \emph{умении видеть}: на улице — улицу; в храме — храм; под злободневною рябью — океанические глубины все те же, что и под спудом мистерий; пиджаки таинственных мантий; и несомненно мне, что та же священная тайна построила формы фараоновой шапки и каски пожарного; в линии орнаментального завитка на дешёвеньком \emph{ситчике} — линии священнейших таинственных знаков; в бегстве от \emph{„улицы”} правды нет: в умении видеть на улице \emph{сфинксовы тайны} — посвящение в мудрость XX века. Собственным ликом на вас смотрит Рамзес II, фараон — из под стекла, в музее среди каирских кварталов, напоминающих мне Париж, а его двойник, полисмен, феллах стилизованным \emph{египетским} жестом поднимает на улице белую полисменскую палочку среди автомобильного тока; подлинный лик Рамзеса — феллахский; между Рамзесом и статуями Рамзеса в Мемфисе — ни малейшего сходства нет: впечатления эпохи Рамзеса посещают вас и на улице; вы бываете странно выбиты — изо всех культур и историй.

\section*{11.}

Кто не знает этого переживания во время горных подъёмов? Вы живёте в маленьком городке; вы охвачены его жизнью; и вы в неё впаяны; вы\marginpar{26} сидите в кафе; и со всеми вместе склоняетесь вы над газетою; переживаете события мировой войны, бродите по бесконечным уличкам, над которыми приподнялся далёкий, всё тот же, пейзаж — точно фон декорации; и всё то же озеро плещет, бросая о берег все те же лимонные корки.

Вот теперь вы уходите в горы.

Смотри же:

— Озеро опустилось под ноги и медленно сжалось; разгладилась его рябь, будто скомканный лист оловянной бумаги отполировала детская ручка; и проступили: глубинные, невыразимые тоны вод; так проясняются глаза человека в минуту задумчивости и изливают лазури, из извечного устремляясь в века; тает так в современности злоба дня; из под пены её наблюдаете вы: в современности тысячелетия прошлого.

Изменяется всё: цвет воды, дома, люди, рельефы; бесконечности каменных домовых квадратов и кубов теперь сжаты глубоко под вами — на чётком мысочке; а, казалось, дальние горы — повытянули свои главы, раздались громадно плечами и перегнулись над озером — прямо к вам: на вас смотрят в упор; то, \emph{чем смотрят} они — не война, не события городка, города, столицы, страны, континента, эпохи, периода времени;\marginpar{27} о современности, понятой в нашем смысле, не может быть и речи; и — тем не менее: \emph{этот взгляд} современен; и эти пространства утёсов злободневно кипят множеством неизливных ручьёв; Ассирии, Вавилоны, Египты кипели своей злободневностью; и — откипали бесследно; а эти летящие струи кипели всё так же; и — \emph{тем же} кипели; кипение этой жизни — не мертвенно; мертвеннее — сиденье в кафе; даже мертвеннее — война; всё текущее остановится в XXV столетии, перенесётся в музеи (если будут музеи); на мыску там, под вами, будут выситься, может быть, пятиугольные здания со странными куполами; а кипенье потоков, взгляд горных громадин — останется тем же всё; \emph{тоже} — вызовет он в душе, что — в этот миг происходит; переживание Ганнибала, может быть, стоявшего здесь, вы узнали теперь — с математической точностью; человек XXV века, вы, Ганнибал и пещерный доисторический человек, пересеклись теперь в одном пункте души; и то, в чём вы все пересеклись, есть вечное; кристаллизация культур, эпох, современностей, и довлеющих дневных злоб выкристаллизовались из подобного мига; вы теперь — не над маленьким швейцарским местечком, а над всеми культурами: бывшими, грядущими, сущими; вы у себя самого; потому что\marginpar{28} вы — в космосе: и космическая картина сознания из под порога сознания городского — она перед вами; в дневнике происшествий перед вами разъялися смыслы; загласили они громовыми воплями серафимов; обратно: с глубинного переживания Заратустры слетает покров: и — покрывало Изиды откинуто. Вы теперь — мощный горн, где потенции всех небывших и бывших культур протекают в расплавленном состоянии; и — погибни история: из души человека, как из вулкана, поднявшися, вытекут все культуры с тайнами и бестайностями, все кристаллизации истин, всё несчётное число их покровов; тут вступаете вы в ту область, где разумение законов и истин не в них, а в самом законе законов, их строющем: в ритме строения, в законе метаморфоз переживаний, мыслей, мод, стилей, линий; здесь сознание перелетает порог, потому что и нет его у сознания; порог сознания сознанию не присущ; порог сознания есть всегда лишь граница, извне застилающая мои кругозоры: не граница зрения, а предмет пред глазами стоящий, как вот эта стена моей комнаты, улицы, застилающей зренье горы; зрение моё видит звёзды; в зрении миллионно-вёрстные дали перемогаю свободно я; в сознании — тоже; надо только выйти из рамок; уметь развивать мускулы, спо\marginpar{29}собные приподымать меня за порог передо мною положенных стен, и мне не присущих.

\section*{12.}

Многие жители городов не покидают города вовсе; а для многих больных представление о пространстве связано с представлением о четырёх жёлтных стенках, вытарчивающих у них за окошком.

По отношения к сознанию — то же; мы все больны параличами воли сознания; оттого-то мы ему искусственно полагаем границу — в пространстве и времени; эта граница в пространстве — стена; и эта граница во времени — злободневность; злободневность и процинциализм нашей жизни из поколения в поколение отпечатали в нас свои представления о границе сознания и о пороге сознания; границ сознанию нет; а пороги сознания побеждаемы — в пространстве и времени.

Ведомы нам дали здёзд; и — непосредственно ведомы; а непосредственное переживание тайн истекших культур — звёзды Индии, Персии, Египта, Халдеи — будто бы нам неведомы; и будто бы: чтоб понять нерв истории надо кануть в пыль музейных архивов; но музейные данные не Египту научат нас; а египетской пыли; сам Египет — он в нас; кровь от крови его, плоть от плоти\marginpar{30} его — мы: следует только твёрдо отставить систему фальшивых порогов, в которых будто бы сознание наше заключено, как в тюрьму; эти „пороги” — есть пыль злободневности: её ненужные отбросы; в своих нервах она — та же древность: которую будто бы она заслоняет, и то же грядущее, которое будто бы и вовсе неведомо нам; оно неведомо в своей „пыли”; и оно с нами — в сути: в ритме душевные глубины обнажены нам бывают порою; надо только суметь произвольно их открывать и описывать явления глубинной жизни сознания; в душе каждого обнажаем колодезь, у которого нет индивидуального дна и которые есть выход одновременно в небо духа и космосов; как по тем же космическим принципам образовались планеты всех солнечных систем всех вселенных — точно так же слагаются и слагались фазы истории; и закон фаз во мне; улови его — история встанет передо мной разоблачённой в её объясняющем стержне; а понимание многообразий мелодий её — только чтение нот; суть не в нотах, а в принципе нотного чтения; и умение чтения знаков событий, искусств, стилей, мод, философий, религий и душевных движений — в умении опускаться в себя, открывая горы и пропасти; кажущееся противоречие между жизнью в себе и в других устраняет первый\marginpar{31} опыт внимания к своим собственным душевным движениям; кажущаяся их хаотичность, невнятность, бесстроица; т.е. всё то, что любители звучных слов именуют \emph{„сублиминальным полем”} сознания — на сознании насевшая пыль: принимаясь за чистку пыли вы пыль поднимаете; но именно поднятие пыли, есть условие очищенья от пыли; при внимательном взгляде внутрь исчезает весь хаос; окружающее отражено вами и чище и чётче; в умении отражать — умение понимать; душа — зеркало мира; замутнено оно — мира нет.

\section*{13.}

Злободневность летит перед нами на своих острейших углах; быт и вкусы кричаще ломаются на протяжении трёх-четырёх поколений; так в малых масштабах линия исторической жизни рисует контрасты нам; а в огромных масштабах господствует закон сходств; многообразие изломов истории в её мыслях, вкусах, костюмах, архитектонике стилей, быта, привычек — образует лёгкую рябь по отношению к основной неизменяемой прямой линии времени; на известной стдии восхождения стирается эта рябь: и отражается в зеркале жизни душа человека; жест куафёра и жесты Фридриха Ницше рази\marginpar{32}тельно противоречат друг другу, пока вы в злободневности, верней в её отбросах; но погружаясь в себя или олько чаще карабкаясь в горы: вы начинаете понимать оба жеста, как модуляции единого душевного жеста; вся бесстроица мимик теперь — метаморфоза немногих нот мимики, ведомых вам изнутри и глядящих извне на вас: улыбкою египетских статуй, улыбкой Джиоконды и…, может быть, вами  виденной булочницы; понимаете вы: что душа древней критянки не так-то закрыта от вас, раз XVIII век, столь вам близкий, повторяет пред вами древнейшие критские моды, как гласят нам раскопки.

Возвращаяся с гор, только чутче уловите вы всё \emph{горное} в долинной деревне; возвращаяся из себя в злободневность, вы увидите — вечность в ней.

\section*{14.}

Ницше, модница, мировая война, дневники происшествий, раскопки на Крите, священные письмена и текущий судебный процесс — злобы дня Вечности, потому что \emph{вечное} — злободневно; умени переносить тысячелетья, в мгновенья, умение переживать мгновенное в вечном — злободневнейший нерв злобы дня, потому что Сфинкс истории стоит перед нами, Эдипами, и предлагает нам\marginpar{33} свои загадки и тайны: злободневными темами, и злободневнейший ответ злобы дня — есть наш ответ Сфинксу.

Наш ответ есть судьба.

\section*{15.}

Человеку пора призадуматься над судьбой человека; и — европеец задумался; человеку пора апеллировать к смыслу жизни, который в нас вписан. Как в бы труп убитого, конкретного мира не вошёл дух иного, \emph{недолжного} мира, противного нашей совести?

Нужно твёрдо нащупать в нас \emph{человечество}; например: не ангельство, не машинность; в настоящее время испуг пред машиною вызывает в иных обращение к религиозным истокам; но возвращение к Богу у многих носит машинный характер: многие защищаются Богом, противополагая его звериному лику природы, противополагая божественность самому человечеству; и обращение к Богу носит характер возвращения вспять.

Мы должны вернуться к истокам великого Гуманизма: возрождение, традиции Возрождения, его широкий, вполне гуманный размах — вот что нужно усвоить нам. Схоластическим прением с бездушной машиною и преданием всей культуры\marginpar{34} костру инквизиции мы ускорим лишь наш печальный конец; отдавая свободное, автономное лицо машине в исканиях нового „Ангельства”, мы приблизим машину к себе; возвращение к средневековым суевериям обернёт мир машин нам в бесовскую рать; возврат суеверий вернёт навождения; и аппарат — будет нам: Господин Аппарат; Господин Аппарат будет новым Hircus Nocturnus'ом на шабашах человечества. В образах и подобиях механической жизни напечатлелые средних веков; механическая культура тесно связана с тёмными сторонами схоластики; схоластика возродилась в \emph{методике}, в \emph{методологии логики}; но она пустила более глубокие корни. К схоластике проведенья понятия по методической шкале присоединилась схоластика проведения нашей жизни по машинному ряду.

Средневековые вкусы новейшего модерниста — естественный дополнительный цвет к… механическому мировоззрению современности; от Рожера Бэкона — к Лойоле; и от Лойолы — к фабричной машине: вот путь нашей жизни. Соединение исповедальни Лойолы с фабричной конторой — сюрприз, ожидающий нас.

Кризисы материальной жизни — колебание её идейно-конкретной подпочвы; идеология — резуль\marginpar{35}тат конкретного творчества; идеология материальной культуры, разлитая в мире, — подлинная причина войны; а война, — выражение внутренно скрытой болезни, глодавшей вселенную: нечто вроде лихорадочной сыпи, проступившей из крови — на коже; тут втиранием мази ничем не поможешь; изменений крови — вот корень лечения; он — в перемене ритма пульсаций.

Наше сердце пульсирует метрами многогромных колёс и метрами дребезжащих трамваев. Пусть же оно запульсирует метрами песен, пусть он осаждается бытом: анемия сознания перестанет изламывать жизнь; механика пресуществится в органику; в противлении человеческой совести механизму — трагедия сознавания; и в ней — кризим; внешние выраженья его — междуусобия, войны, болезни, убийства; и, наконец, — помешательство.

Механическое прекращение нас обставшего кризиса будет безплодным втиранием мазей — лечением прокажённой кожи культуры вместо лечения заболевшего сердца.

Грохоты горизонтов сознания не прекратятся: они — это — мы; мировая война — alter ego.

\section*{16.}

Вот сейчас я пишу, а „гром” — непрерывен: глухой говор поднимется; глухой говор потом\marginpar{36} упадёт; а сегодня — многие говоры, перебегая друг в друга, сливаются в басовую, глухую и тяготящую дрожь:

— Уу-у-ууу…

Так гремело всю зиму, всю осень, всё лето, всю весну, всю прошлую зиму, всю прошлую осень и валились прохожие между грудами черепитчатых домиков; и валились из окон, из груды перин, и — уставляясь в закат… с выражением, точно из них выпирало сплошное, тупое, больное, „оно” — огромное, неживое какое-то.

\begin{Verbatim}
Ты подвиг свой свершила прежде тела —
Безумная душа…
Под веяньем возвратных сновидений
Ты дремлешь; а \emph{оно}
\emph{Бессмысленно глядит}, как утро встанет,
Без нужды ночь сменя;
Как в мрак ночной бесследно вечер канет, —
Венец пустого дня… —
\end{Verbatim}

\hspace{0.45\textwidth}— говорит Барытынский, — и вот \emph{„оно”}, это тело, говором пушек утомлённой души глядит \emph{„оно”}… в листики местной газеты, оповещающей мир и постановке нового фонаря между Дорнахом и Арлесгеймом; и — о битвах в Европе.

Глядя на жизнь, распростёртую передо мной в тихих сёлах, мне кажется, что действитель\marginpar{37}ность рухнула, как дебелое тело швейцарца, в перины: рассыпалась в атомах; пляска атомов сонного тела не жизнь; это — казнь; это — месть за изъятие мира из мысли; мир, изъятый из мысли, — ни мир и ни мысль; он — не он, не она, а какое-то неживое \emph{„оно”}, безучастно вперенное в нас и свершившее подвиг свой; такой взгляд — тихий взгляд помешательства.

Не странно ли: до-военная суматоха напоминала сплошной „тихий час”; обыкновенные грохоты жизни как будто теряли свой голос: и настала она — тишина; иль вернее — \emph{„оно”}, — то „оно”, что уже девятнадцатый месяц томит меня здесь в Базельланде.

\section*{17.}

Тишина русских ширей — прозрачная, ясная, внятная, окрылённая в грусти своей: тишина ожидания.

В моём тихом углу бесстремительно всё; каменно тишина припадает: поникаете вы у себя на дому; поникаете в ожесточённо рыдающем ветре; поникаете в летнем безветрии; и все — точно валятся на дорогах меж грудами черепитчатых домиков, друг другу подобных; валятся и молчат: тяжело говорят, тяжело глядят исподлобья; и — тяжело поступают.\marginpar{38}

Тишина Базельланда — \emph{„оно”}; про \emph{„оно”} поют петухи; и отбивает отчётливый колокол — всё про то, про одно: про \emph{„оно”}; и \emph{„оно”} здесь во всех; и \emph{„оно”} здесь во всём; и все — в „нём”.

По ночам „оно” почивает в перинах, не слушая, как тихо дзынкнет окошко, когда… громыхнёт с горизонта.

Отяжелевший, бессмысленный, неосмысленный мир!

— И тяжёлая мысль, потерявшая смысл!

\section*{18.}

Ни одного события!

Всё — мертво; всё — спокойно; издали лишь военный оркестр вечерами наигрывает грациозные вечерние зори: прекратилась стремительность суетливых движений военных отрядов; изредка, загорелая кучка солдат пробушует на улицах деревушки; и — пропоётся тут песня; разговоры о том, что, мол, выстрел — убитые жизни, — разговоров таких теперь нет (попривыкли мы к „говору” пушек); говорится — о керосине, об угле, о сахаре: подорожали продукты; подорожают ещё: то ли будет!

— „Ру-рууу” громыхнёт с горизонта.

Тянутся сонливые дни; и проходят бессонные ночи; никогда ещё меч сомнения не рассекал\marginpar{39} наши души такою огромною болью; никогда ещё разделение не секло нас этой секущею силою; кризис сознания пересекается с кризисом жизни; были же голоса — отчего их не слушали?

Я в туманную ночь открываю окно; из туманной ночи вылетают снопы: световое пятно загорится на тучах; разбегается, угомониться не хочет: сигналы.

Затворяю окно: продолжаю писать.

\section*{19.}

Связь вещей — в моём «Я»; эта связь есть сознание; «знания» — члены связи — неизменяемы: изменяема комбинация их; она — в ритме; ритм знаний — сознание; в модуляциях протекает оно, оплодотворяя нам знания; в оплодотворениях творчески раскрывается наше «Я»; все понятия знаний рассудочны; вне их вяжущей связи они рубят действительность на бессвязно-текущие части; в переложениях и сочетаниях, во взаимном их контрапункте — в сознании — окрыляется разум; в сознании — цельность «Я»; оно — знание собственно; оно — родит свои формы; в оформлении, в \emph{«ставшем»} нет жизни; в \emph{становлении} строится \emph{ставшее}; в ритме строится форма; в сознании — знания; сознание — центр текущего организма вселенной; вне его организм этот — труп.\marginpar{40}

Из зерна бежит стебель; из стебля наливается колос, а в колосе — зёрна; и — так далее, далее; \emph{прежде} и \emph{после} — отсутствуют; отношение сознания к миру и мира к сознанию — отношение колоса и зерна; из раздельность — абстракция; мысли мира и мыслимый мир суть единство в сознании; задрожи оно — рухнет мир.

С нами кризис сознания; и — стало быть: кризис мира; земля — в знамениях; стало быть: будут знамения в небе; твердь земли и небес — она дрогнула в нас; сознание наше мутится: диссоциация материи налицо; о ней кричит физика; и — диссоциацию духа являют нам наши вкусы — в искусстве и в жизни; футуризмы, кубизмы нам убивают искусства; вырождение, кретинизм убивают нам нашу жизнь.

И пора нам спросить со всею ответственной строгостью: что есть знание? Чем должно оно быть? И — каково оно в \emph{«знаниях»}?

\section*{20.}

Знание — брак «Я» и мира; \emph{«и познал жену»}… говорится нам в Библии; знание — в слиянии с узнаваемым.

Знание цветка обнимает знание его функцией, тычинок, и имя, произнесённое на звучной латыни; кроме того: предполагает умение пережи\marginpar{41}вать себя в нём; полевую лилию знать — значит стать: полевою лилией в поле; видеть солнце, как лилия; узнать лилию в спиртовом препарате, это значит — придти к убеждению, что она — пахнет спиртом.

Всякое абстрактное знание нас ведёт к суррогату: к лилии… на бумаге, к рисунку и к схеме; такой лилии — нет; её надо доказывать.

Доказательства Божьего бытия — суррогат жизни в Боге; умер Бог в сердце нашем; и мы начинаем — вести разговоры о Боге.

Пусть на полюсе пальма — мечта: она \emph{есть} в жарких странах; полюс ведает лёд; для него всё иное — фантазии; но вера в фантазию, в пальму, покоится на возможности наблюдения пальмы в Египте; отношения \emph{знаний и вер} — отношения египтян к эскимосам; на экваторе снег мифичен; грозовая туча — миф полюcа; предмет \emph{знаний и вер} переменчив; география сознания нашего — она неизменна.

И абстрактна граница меж верой и знанием; невозможны без опыта ни та, ни другое; вера в знании — есть; и есть знание веры: дела веры в опытах; мертва вера без дел, мертво знание без опыта.

Принцип нужен для опыта; вне его он —\marginpar{42} абстрактен; знание без веры абстрактно; и вера без знания — сказка.

Рассудочно рассечение жизни абстракцией принципа; действительность раскрошена ею в формах; материализм есть абстрактное \emph{крошево} трупа мира на мельчайшие части; спиритуализм вылагает принцип из жизни: и уплотняет его — в неподвижностях догмата; и — \emph{каркас} духа — догмат — без плоти и духа: таковой \emph{каркас} — пуст; догматический спиритуализм — вытравление плода жизни из жизни; материализм — убивание самой жизни; разделённые вера и знание, нам грозят оскоплением и разложением жизни; разделённые, вера и знание нам гласят, что \emph{духовное знание} невозможно: непостижен де дух, бездуховно де знание; и, молясь, неизвестности, мы живём в мертвечине; остаются нам знания неверны; остаются нам веры незнаемы.

Мы хотим умных вер, волим верное знание.

\section*{21.}

Доказуемость бытия — стала нам бытием.

Бытие распластано перед нами многообразием научных провинций и континентами методов: бытие стало логикой; и мы пригнаны к полюсу, где торжественно мы стоим, облечённые в ледяной футляр логической формы. Человек шёл и…\marginpar{43} стал: человеком в футляре; зажил он в отложениях: в формах, в футлярах, в каркасах; вне нас — твёрдые, материальные льды: в нас же — мёртвые формальные догматы; так мы зажили — в трупах трупы: сколько прожили так мы, не знаю; но мы сдвинулись с места, мы двинулись… к пушкам: коросты оказались на нас разбиваемыми только пушкой; но с коростом отбиваема наша жизнь; коросты мы сдираем с души вместе с костями и мускулами; разорвалось сознание наше: разрываются вместе с ним наши души, тела; распадается вместе с нами, стареет планета; передвинулись зимы, осени, весны; и — тусклы закаты; землетрясение бежит под землёй на своих гремящих толчках; аритмии чугунного грохота раскатались от моря до моря.

Вот что сделали паразиты — абстракции — в мирах жизни нашей; благополучно мы сидели в уютных квадратах и — квадратами думали, созерцая кубы домов угасающим зрением крыловской мартышки и обложившись десятком очков, в которые мы воистину не смотрели и которые… нюхали…

\section*{22.}

Полное знанье в слиянии мира и мысли, т. е. в связи предметов и знаний: в \emph{сознании нашем}; перестали мы мыслить миром; и мир пе\marginpar{44}рестал нами мыслить; многообразие диалектических сочетаний, всё творчество мысли — бессильно распалось на мёртвые методы и фанатически заострилося в догматах; легконогая диалектика, танец догматов, обернулась в мозгу у нас суетливо-мышиной грызнёю: переживанием неврастеника; так червивыми ходами источили нам мозг мысли наши.

Между тем, диалектика есть произрастание из зерна ветром зыблемых колосьев из мысли и созревание в колосе новых зёрен — творимых действительностей; иссякновения смысла жизни в ней нет; в ней кипение, превращение, наростание, размножение творчества: в смысле смыслов; смысл — многоветвистое дерево; но абстракция его — палка; да, мы древо познания обернули рассудочной, принципиальной палкою; гармонический шелест кроны замолк; раздавались и падали вокруг нас — палочные удары тенденций; и гармония сфер разрешалась надолго для нас в барабанные трески пустейшей словесности.

\section*{23.}

Что такое рука? Это знает лишь тот, кто владеет рукою; знает он, что рука — его орган душевного выражения; и вовсе она не „конечность”, как гласят анатомии; изучение сокращения му\marginpar{45}скула с очень громким латинским названьем к знанию руки не приводит; изучение это совершается главным образом лишь на трупе; труп руки — не рука; рука — в жестах, а в трупе нет жеста; в нём есть содрогание, производимое при помощи электричества; анатомия и физиология рук в лучшем случае научает нас механической дрожи; и действительность этого знания — кинематограф; знание руки — жест Айседоры Дункан; и мне этого жеста не даст изучение трупа; я его увижу в движении пальца Крестителя у Леонардо-да-Винчи; Леонардо-да-Винчи и Дункан — они руку знали; естествоиспытатели рук не знают: они знаю… — \emph{„конечности”}, принадлежащие не человеку, а той же крыловской мартышке; по образу и подобию её мы построили нашу жизнь.

Слишком много есть трупного в нашем знании жизни; красота передёрнулась в ней потрясающим „мартышкинским” выраженьем, напоминающим… агонию и судорогу под вивесекционным скальпелем метода; хирургические ножи — обагрили нам жизнь. Методы хороши у стола оператора, но не в жизни; ни даже… в брани; помните, у Толстого: „Die erse Kolonne marschiert, die zweite Kilonne marschiert”. Методика, догматизм нам присущи скорей: мы больны методизмом и да, мы абстрактны; оттого-то духов\marginpar{46}ные блага, живущие в нас, в нас порою иссушены нами же; отвлечённость наш враг; вера в методику — наше хамство пред немцами; ведь и метод живёт в модуляциях метода; если метод повесить на стенку, он — мёртв; вера в метод нас давит; наоборот: модуляция методов, принципов, догматов — разбивает на методе методический корост; такой метод есть ритм, есть пульсация живой жизни; самый метод есть тень ритма жизни; мы по тени должны отгадать её подлинный лик; мы же тянемся к тени, а всякая тень — опрокинута; и проклятие нашего представленья об истинах знаний — есть жизнь в опрокинутых истинах, в истинах \emph{„вверх ногами”}; Истина \emph{„вверх ногами”} гласит, что она-де — абстракция; участь наша есть участь Пилата: теоретически истину вопрошать, что есть истина; Истина не отвечала Пилату: истины не призваны отвечать на вопросы об истине; истины предстают, чтобы их видели; за истиной следуют без вопросов.

\section*{24.}

Жизнь течёт в быстрых жестах; молниеносны вопросы; молниеносны ответы; знание цельное — жестикуляция и ритмический дар; знание — импровизация среди случайностей жизни; дар — в\marginpar{47} единстве сознания, в верной связи конгломератов узнаний; эти узнания — ноты; умение слагать песни из из нот — в этом корень сознания.

Методически можно, конечно, исчислить и грацию эстрадного танца в механических формулах; исчисление будет длиться года; можно мгновенно дать грацию в танце; цельное знание есть мгновенный подарок; в нём ответы на мимику восстающих вопросов; теоретический ответ отстаёт; или он пустая абстракция ограниченного „принциписта”; принципиальных и полных ответов на кризис сознания ждать нам некогда (мы прождём их столетия); и остаётся ответить мимикой и ритмами действий — по дару.

События жизни взывают к мгновенным ответам, не к отвлечённому знанию, а к сознанию нашему; в умении разговаривать с фактом — оно; и оно в умении называть имя факта; нет его в арсеналах \emph{знаний в ковычках}, в дровяном складе догматов, в археологическом музее культур, где собраны части когда-то конкретной действительности; ныне брёвнами „принципов” не разрушить темницу сознания нашего; ныне „принципы” — ритмы духа: и он дышет, где хочет — к огорчению методолога; методология — поражённый миной дредноут; переборки его закроют пробоины; некогда нам „принципиально”\marginpar{48} трудиться: опускать переборки и обстругивать догматы. Нам события обнаружили ясно: мы жили на полюсе, где забыли мы грозы; громовые удары на полюсе — мифы.

Нет, единственный выход сознанию нашему: принять правду грозы; сбросить веру без знания; искать веры своей, познаваемой точно; искать \emph{верного} знания. А пока мы будем седлать отжитые веры и догматы, мы останемся с арсеналами непререкаемых теоретических положений без… твёрдого положения в мире; так способны мы выехать, точно дети на палочке, из тропической разрешающей атмосферу грозы; так способны вторично мы выехать в пространстве полярных абстракций, чтобы там с самоедами утвердить былой догмат: Бог, гроза и цветок — только миф египтянина.

Помним: догмат — дредноут; действительность — подводная лодка; мина — факт: незабываемый факт.

Ах, побольше бы мимики, ритма, жизни, свободы движений и мысли, конкретности, правды: меньше, меньше очков; что в них проку? ведь мы их не носим… почтительно нюхаем, уподобляясь крыловской мартышке; мы глядим на очки, не сквозь них; так не видим мы фактов в многообразии фактов и в неожиданном\marginpar{48} опыте — кризис сознания наступает; бьют часы… время — действий.

\section*{25.}

Помню, как оживленно здесь обсуждалися телеграммы, как спорили; раздавались задорные голоса, взрывы смеха в разгар шумных споров; спорили добродушно: в войну нам не верилось.

И вот — она грянула.

Это было под вечер: закат был багряный; пророзовели верхушки Эльзаса (кто мог думать тогда, что оттуда покатятся к нам гремящие звуки?); помню — весть: мировая война разразилась; руки стареньких поселянок протягивались к чуть белеющей гряде гор, подымающейся за Рейном, из Бадена; там оттуда-де — пушки: всё в пушках!

Прошло десять месяцев: всё водворилось на место; ещё изредка трещал барабан; сытые, ленивые головы повылезали, как прежде, из окон; и — из-за груды перин, переговариваясь с такими же, как они, головами, глядящими из окошек; а — „громыхало” всё громче, всё громче, с белесоватых холмов — там за Рейном! — из Бадена, где все пушки: глядели, казалось, на нас.\marginpar{50}

Пели мирные петухи; колокола звонили, к полуночи; с десяти часов вечера — ни души, ни огня!

Не думаю, что провидят односельчане мою космическую сверх-размерность войны; даже… кризис сознания, не переплетаемый вовсе ни в белую, ни в оранжевую, ни в голубую, ни в красную книгу: переплётчику войны не отдашь; семи красками спектра её не окрасишь; не разъяснишь её ни порывами благороднейших, или даже негоднейших чувств; в классовое сознание тоже не втиснешь.

Даже базельцы не поверили: ни одному переплёту войны; предпочитают её не понять, чем понять однобоко; в этом есть своя правда; в этой правде своей они ходят давно; и — молчат, посасывая короткия трубочки.

Все увидели мы, что из бури свинцового грохота не рука человека грозится руке человека; и — поднимает тот грохот.

\section*{26.}

Но ко всему привыкаешь…

И мы совершаем поездки — из под базельской деревушки: послушать Бетховена; но и над Базелем она — гремящая тишина.

Она — вошла в восприятие… ну, как пение комаров, визги ласточек, визг далёких трам\marginpar{51}ваев, или — шум водопадика; и умолк он, тут-то вы и заметите, что он — был; и когда гремящая тишина неожиданно станет безгромной, то обитатели Дорнаха говорят:

— „Слушайте: перестали стрелять!”

Вот закаты здесь хороши: лиловобагряные тучи несутся — клочкастые, быстрые: по бледно-зелёному небу; и оно — всё горит: рдеет кровью, может быть, пролитою вот только что, — в пятнадцати километрах от нас; в этой гаснущей рьяности — блистательный треугольник из двух немигающих звёзд и юно-хрупкой полоски серпа полумесяца; огромные две звезды соединились так близко; Юпитер с Венерой — с любовию мудрость, и над Эльзасом стоит: соединение в небе; но на земле — разделение.

Мёртвая данность распалась; и уж земля — не земля: распадается наша зямля; человечество отравилось субстанцией кометных хвостов; головные абстракции привлекают на головы наши комету.

\section*{27.}

Тихими вечерами блуждаю по мирным долинам и вспоминю: далёких знакомых; помнится один вечер — с той поры прошёл скоро год; на стеклянеющем небе поднялся какой-то предмет; и, розовея, повис в неподвижности; я на\marginpar{52} него засмотрелся; он же — стал розовым облачком; нежное, оно вскоре истаяло: а предмета уж не было; но рассказывали газеты, чем был тот „предмет”; там шёл бой — столкновение цеппелина с аэропланами, кажется; кажется, там кого-то кто-то расшиб (мы не слышали выстрелов); но я видел: на стеклянеющем розовом небе поднялся предмет; и стал после — облачком; облачко порастаяло; а куда девался предмет?

Это было уже — скоро год; но всё так же \emph{„гремит”} с горизонта.

Из-за перины \emph{„оно”}, сонливо глядящее тело, с острия всей культуры — всё так же, всё тоже — уткнулось в окошке глазами в газетку; и — вероятно читает: о починке поставленных фонарей меж двумя деревушками.

\section*{28.}

Слушаю глухой говор орудий с эльзасской границы; и почему-то мне кажется: глухой говор знаком. В глубине деревенских полей подымался он некогда; перепелиные крики стояли: из-за ржи, в васильках, — кто-то всё подъезжал; явственно громыхала телега — далеко, бессменно: по вечерам на зоре. Бородатый лесничий, мой друг, на приступочке белого домика исподлобья, бывало, посмотрит; и — спросит, бывало:\marginpar{53}

— „Ведь… едет?.. Ведь… едет же?”

Перепелиные крики стояли; и — \emph{ехало}, не доезжая, — там, из-за ржи: в васильках; гремела — телега ли, пушка ли? Это было под Луцком, в 1911-ом году.

\emph{Подъехало}: кажется, белый домик разрушен уже — орудийным говором.

Глухой говор гремел над ночною Москвой — тому назад десят лет, над апельсинником италийской долины; и — в ковылях русской степи; я ждал, что он грянет.

Он — грянул уже.

Дни — текут… Времена накопляются… Приближаются поступи сроков… и — исполняются сроки…

Но разве не помните вы? Про 913-ый год говорилось так много в крестьянстве: на год просчитало крестьянство; слышало и оно — глухой говор событий, как его расслышал поэт:

\begin{Verbatim}
Опять над полем Куликовым
Взошла и расточилась мгла,
И, словно облаком суровым,
Грядущий день заволокла.
За тишиною непробудной,
за разливающейся мглой, —
Не слышно грома битвы чудной,
Не видно молньи боевой.
Но узнаю тебя, начало суровых и мятежных дней…
                                                                      \emph{А. Блок.}
\end{Verbatim}
\marginpar{54}

Не над Россией гремело: гремело в Европе, гремело над миром; гремит и доселе — за громами пушек: \emph{грядущие громы}…

Слушаю глухой говор орудий с эльзасской границы; неудивляюсь ему; уезжаю в Базель гулять и подолгу стою над зелёными струями Рейна: над струями — чайки.

Каждый знает минуты невнятицы: я люблю подслушивать глухонемую невнятицу — полудрём, полумыслей: надо и их выговаривать членораздельно и внятно.

\begin{Verbatim}
Парки бабье лепетанье,
Жизни мышья беготня…
\emph{Я понять тебя хочу} —
\emph{Тёмный твой язык учу}.
                                        \emph{Пушкин.}
\end{Verbatim}

\section*{29.}

Удаленье от точного смысла в детали из предварительных изысканий построило нам лабиринты, в которых запутались мы; средства стали нам целью, когда эти средства мы сделали средствами техники; и при помощи техники строили храм из машин; появились философы, обосновавшие эту подмену; машина нам стала воистину: \emph{„объясненьем в себе”}; цель её — в средствах; и средства в ней — цели; хитрое понятие \emph{„целе-причинности”} изготовил нам Вундт;\marginpar{55} сто ловкое идейное шулерство в своё время подобострастно глотали мы; и старались свой мозг приспособить к „целепричинной” действительности; \emph{„целепричинная”} жизнь привела нас к борьбе: разве нынешняя немецкая поговорка „Not hat Keit Gebot” не оправдана „Метафизикой” Вундта и философией „Als ob” Файгингера.

Мы зажили — \emph{„прагматически”}: приложили понятие цели к бесцельно бурлящему чреву.

Парализовано внимание к \emph{чистой мысли}; и оттого перепуталось всё; цель со средствами смешаны; в целях средства оправданы; в средствах цели даны.

Удаление мысли от целей познания переживалось сперва романтически, взрывами энтузиазма и праздниками освобождения якобы \emph{„мысли”}; в этих праздниках превращения познавательных средств в цель науки протёк весь XVII век; и — протёк XVIII век.

Дифференциальное исчисление оказалось приложенным… к пушке: Лейбниц, Ньютон, Декарт поступили на службу к „солидному” Круппу; сорока-восьмифунтовое орудие прокричало учёным:

— „Виват!”

И во славу науки оно принялось вдруг выкидывать \emph{„чемоданы”} над башнями храмов.\marginpar{56}

\section*{30.}

Органический смысл моего бытия в том, что „Я” — неизменен и вечен: \emph{sub specie aeternitatis} живу; этим „species” были в начале пронизаны действия.

Теперь „species” в биологии есть — отбор: отбор особей; он меня аннулирует; „я” есмь „я” лишь постольку, поскольку я исполняю свои детородные функции.

„Species” смысла нам дан социологией; понятия общественных механизмов меня аннулируют; смыслы „я” в содействии механической сумма из „я”, шествующей до ближайшей канавы, — дойти до канавы (биологически смерть есть канава); мы живём для того, чтоб пробег до канавы \emph{„детей”} совершился бы комфортабельней; но пробеги тысячей поколений до ближней канавы (до смерти) свершался уже: все попали в канаву; все сгнили в канаве; дорога — утоптана; что же?

В итоге пробегов — дорога в колдобинах: все миазмы болезней, вся грязь нищеты и разврата теперь заливают колдобины; колдобины превратились в траншеи; в траншеях устроились — жить: обзавелись фортопьяно, литературой, вином; может быть, человечество, не желая бе\marginpar{57}жать до канавы, устраивает себе канаву искусственно; и мы у преддверия новой жизни — \emph{„канавной”} (траншея — канава)?

Посмотрим.

Колдобины превратились в траншеи; на рёбрах поставили пушки; сидят и стреляют сорокавосьмидюймовыми чемоданами, отправляясь от Лейбница и Декарта, рекомендованных Круппу почтеннейшим Вундтом.

Дорога испорчена: мировая канава — конечная цель! — есть \emph{Ничто}.

Достаточно быть знакомым с историей, чтобы раз навсегда отказаться от смысла, коль смысл нашей жизни — канава; и во вторых: наш пробег до канавы не улучшает дорогу, но — портит дорогу.

\section*{31.}

И философия лени невольно встаёт: мы философы лени — „канавные” жители! — разглагольствуем с правом теперь: —

— Не побоимся лениться! „Служенье муз не терпит суеты!” Дел, настоящих дел, у нас нет: быть не может; все дела — золотой фонд богатств — подменили бумажками мы \emph{„патриотических”} военных займов; богатства страны обернули мы в залежи динамита и мелинита; а\marginpar{58} их выпускаем мы в воздух, верую в основной закон физики: в круговращенье энергий; и позабыв, что закон этот в мире Гельмгольца и Томсона ограничивается законом рассеянья и ростом таинственной \emph{„я”}, грозящей нас всех навсегда обанкротить; мы развеяли силы, богатства и жизни два года в космические пространства вселенной, в наивности полагая, что из пространства осядут на нас \emph{„великие и богатые милости”} в виде яств, утучнённых тельцов и согревающих тканей.

\section*{32.}

Если мы будем дальше так жить, вижу явственно я: лень, апатия, мертвенность — предстоящая нам девальвация; ибо темп — темп войны — нас обрёк на безделье в грядущем. Тепловую энергию жизни, жар жизни, ухлопали мы почтеннейшими хлопушками в виде сорокавосьмидюймовых орудий.

Все дела „обезделились”, обессмыслились; превратились в обычную деловую сериозную суету; суету возвели мы в квадрат; \emph{суета сует} — жизнь Европы; по отношению к ней не мешает нам погрузиться теперь в философию „Экклезиаста”:

— \emph{„Суета сует, сказал Экклезиаст, суета сует, — всё суета! Что пользы человеку от всех трудов его, которыми трудится он под солнцем? Род проходит, и род приходит, а земля пребывает во веки. Восходит солнце, и заходит солнце, и спешит к месту своему, где оно восходит. Идёт ветер к югу, и переходит к северу, кружится, кружится на ходу своём, и возвращается на круга свои… Что было, то и будет; и что делалось, то и будет делаться, — и нет ничего нового под солнцем”}\footnote{Экклезиаст: I 2–6, 9.}.

Лишь когда мы ленивы, порой заживают в нас истинно бескорыстные мысли; всё прочее — утилитарно; медитация в условиях нашей суетной жизни приходит свежительней через лень.

Медитация — лень, возведённая в принцип; царство лени, — корабль, отплывающий в страны кипений бесцельности; будемте тише, ленивей и вспомним пословицу: „Тише едешь, дальше будешь”.

Суета сует, выростающая из „деловой”, „трезвой” жизни есть мысль, что — „я”, такой, каким несу себя через жизнь — не мертвец, что ещё\marginpar{60} не всё погибло, не все пути отрезаны к возрождению.

Оставимте компромиссы религий и мистик; они — наши \emph{тати}; от них содрогаюсь надеждою „я”; но содрогаюсь надеждой не „я” — труп во мне; мои черви во мне копошатся (то — нервы мои) и кричат:

— „Живы мы! Ещё есть нам спасение!”

Я — погиб безвозвратно; погибли мы все; и не будемте гальванизировать наши трупы; моя кожа давно мною сброшена (вместе с природой, откуда я выпал); мои обнажённые нервы — суть черви, давно источившие тушу мою; моё мясо, пронзённое нервами, напоминает одежду, покрытую паразитами: мои нервы кусаются; жизнь их — адская боль для меня.

Изящен во мне лишь скелет; в нём — „бессмертие” смерти; через бессмертие смерти душевной пути нас ведут к возрождению духа.

\section*{33.}

Ужасны глаза мои; голубые они — от разложившейся крови, фосфорический блеск их — продукт разложения; одушевление разлагает меня; мои блески в глазах — просто гниль! А движенье зрачков — только чёрненькие головки двух трупных червей — долей мозга — заползших в\marginpar{61} глазницы; когда-нибудь эти черви сожрут содержимое тёмных глазниц; темнолонные впадины черепа обнаружатся явственно.

Мы дети — Каина: наши пути ведут к гибели; да, кто-нибудь, из погибнувших и воскреснет, быть может; строить мысли о том, что воскресну „я” именно, — значит длить агонию: (\emph{тридневе есмь}: и — смердит); не хочу агонией питать в себе стаю червивую „нервов”; \emph{„темперамент”} мой их питает; не убиваемы черви во мне; погибают они лишь от голоду; уморить бы мне их моей смертью!

Меньше трепета, одушевленья, надежд, блеска глаз и градацией „интимнейших” жестов: побольше суровости; „интимности” — показная личина червя; подлинный \emph{interieur} есть скелет.

Я есмь труп. Никаких утешений не надо: утешением будет мне мысль: \emph{утешения нет} — безутешен. Пока тело, сгноенное мною на мне, ещё носится мною, утешение мне — безутешность моя.

Бескорыстие высекается лишь могилою.

Я — погиб безвозвратно.

Вот — единственная философия, нам способная указать пути выхода из тупика. Нашей жизни, — „канавы”, „в которую мы залезли, которую рыли” себе столько лет, вопреки голосам, преду\marginpar{62}преждающим нас о близости катастрофы; лучше вовремя нам черпать силы в суровости, чем воскликнуть, как Иов: —

— „Погибни день, в который я родился, и ночь, в которую сказано: зачался человек. День тот да будет тьмою… Да омрачит его тьма и тень смертная, да обложит его туча, да страшатся его, как палящего знаю… Для чего не умер я, выходя из утробы…? Зачем приняли меня колена? Зачем было мне сосать сосцы?… Вздохи мои предупреждают хлеб мой, и стоны мои льются, как вода, ибо ужасное, чего я ужасался, то и постигло меня; и чего я боялся, то и пришло ко мне”.\footnote{Книга Иова. Гл. III.}

\section*{34.}

В Базеле созревала мысль Ницше. Его любил Беклин. Здесь профессорствовал много лет достойнейший Яков Бергхардт; и жил некогда математик Бернулли; действовали — и Эразм, и Гольбейн; дом Эразма сереет доселе: средь безлюднейшей улицы; временами живали поблизости: Грюневальд и Неттесгеймский Агриппа.

Таково созвездие ярких, славных имён, восходивших над Базелем; оно связано с Возрож\marginpar{63}дением и эпохою великого Гуманизма; в это славное прошлое подымается тихий Базель розоватыми и серошершавыми башнями; и серые замки торчат здесь осколками; они сидят на Юре, на лесистых отрогах Шварцвальда, на гребнях Эльзаса.

Многие битвы шумели над Безелем; первое поражение рыцарей швейцарскими мужиками произошло здесь поблизости; ныне над грудою черепов подымает крест свой часовня.

Сходятся здесь и Шварцвальд, и Юра, и Эльзас; сам же Базель в долине; холмики избороздили её; с холмиков поднимается он; грудами черепитчатых домиков; серо-розовым Мюнстером и ярко-красною Ратушей привстаёт он над Рейном; веснами лиловеют в гирлянде гладиний его серобокие домики; магнолия зацветает в садах; осенями и зимами он дымится в туманах; по нём бегают глянцы неизливных дождей; всюду обилие очень старых, каменных, крашеных, полноводных бассейнов, поднимают статуи столбики — золотого рыцаря, гражданина в заломленной шляпе, прелатика, или просто дракончика, изрыгающего струйку чистой воды.

Улички здесь горбаты и кривы; тусклые фонари на стенах; и малоглазые домики нависают —\marginpar{64} полосато пёстрыми выступами; бедно одетые кучечки соберутся под выступом; неподвижно медлительные люди; они сосут трубки; и — провожают вас взглядами; из окошка порой вы увидите — колпак старика; и он — жуёт трубку; выпирает из шеи его [очень часто зобатой] густой, белый войлок; непременно покажется вам, что оконная рама есть рама портретов Гольбейна, которых вы видели в великолепной ли базельской галлерее, в книгохранилище ли, где работать отрадно на старике был берет; и рисовался на фоне он из голубоватых и бледнозеленных материй.

Этого старика вы увидите: в котелке, в кососкроенном пиждаке, с дымнокудрой сигарой в руке в более молодых частях Базеля; он покажется жалок там.

\section*{35.}

Неприязненно отбегает новый Базель от Рейна: грудами невысоких, торговых домов и кубами возводимых построек; хорохорится суетливой гримасой немецкой провинции его банки, таверны и лавки с характерными надписями вроде „Тысяча брюк”; раздувается в огромный вокзал; и желтеет от скуки.

Этот Базель напоминает мне толстого буржуа, буржуа собираются в „Казино”; поднимают там\marginpar{65} горластые дымогары; и тупо тычут в шары биллиардными киями; а улицы — суетятся; людоход непрерывен; в говоре голосов временами прорежется — глухой говор орудий с эльзасской границы.

Если бы вдруг в толпе перепутать носы, глаза, спины, руки и далее — самые цвета тканей в безотраднонелепые сочетания, то получилось бы точная копия базельской уличной жизни, где все платья повисли, подбородки враждуют с усами, а ритмы рук — с ритмом ног.

Точно сотрясся телесный состав человека; и оставшийся кавардак — базельский буржуа. В нём типичное стёрто; оно не стало немецким; базелец стоит одиноко; он — мозаика, выцветающая от времени; и потому — однотонная.

Однотонность безстроицы — канва впечатлений; вненациональное не достигло размаха городских, больших центров; а местное — стёрто; острокрылатого слова нет; и скрипучая пересыпь слов, очень громких, гортанных, будто тренье кремней друг о друга, перетирает в кафе все газетные сведения.

Базелец не доверяет газетам и ходит в концерты на громозвучных певцов — басов, теноров, баритонов и \emph{„деритонов”}; любит он оркестр барабанщиков.

Оркестр барабанщиков — украшение Базеля.\marginpar{66}

Проживя здесь так долго, дошёл до того я однажды, что принял участье в оркестре любителей: и — играл на втором барабане.

\section*{36.}

Слово базельца напоминает скрипенье кремня, а не порхание бабочек: бабочка стрясает пыльцу; из кремня летят искры: искрами горячего гуманизма и огромной волей к добру загорелся он в дни войны; странное сочетание он из примитивнейших предрассудков и очень тонкого такта.

В базельце подчеркнулася не простая обязанность быть корректным со всеми, а активная воля быть подлинно человеком; развитие социального такта сказалось в проявленной мягкости — ну хотя бы ко мне. Я тем более ценю этот такт, что население по интересам и связям естественно тяготеет к Германии.

Базельский обыватель — наполовину в Германии; в нескольких километрах лишь Баден; и на несколько километров подальше — Эльзас; население у границы смешалось, и близость к Германии напечаталась на мелочах здешней жизни: вы здесь встретите и пивную из Мюнхена, и германскую лавочку, и чиновника в характерной прусской фуражке.

В Базеле — немецкая таможня.\marginpar{67}

Я на днях постоял в десяти шагах от Германии. Я смотрел на открытую жизнь по ту сторону немецкой границы.

Крутобокие горы пушились набухшими почками; одуванчики зацветали; и оснежались долины цветущими вишнями; переговаривались желтосерые ландштурмисты; и поглядывали на нас, держа ружья под мышками; по Германии бежал поездок; и тихая вилла глядела с холма…

С нетерпением подъезжал я к границе: из-за цветов и кустов побежали на „трам” миловидные, весенние виллочки; и уже вот он, вот: набегающий Базель — с кубами серожёлтых торговых домов и с гортанно-вещающим людоходом; Рейн и серые башенки показались и скрылись; вот прополз зарейнский рабочий квартал; показались немецкие лица; показалось обилие неизбежных пивных: зарейнская часть стремительно переходит в границу.

Вот — построенный на швейцарской земле баденский, немецкий вокзал: широчайшее помещение в тяжеловесных колоннах и глотавшее поезда, и плевавшее поездами; оборвался ток товаров; с войною артерия эта перерезана здесь; запертой вокзал пуст; на огромном перроне — никчёмная кучка: характерные прусские картузы с высоко приподнятым краем.\marginpar{68}

Кончился Базель: поле…

Прожелтилось, проголубело цветами оно; солнцем и травами засмеялась долина и уткнулася прямо в горы; это горы — Шварцвальда.

Вот — белая деревушка, вот — церковь: отчётливы; перебежать бы луг да на холмик! Нельзя это — Баден; перебежишь — и не вернёшься обратно: ты — пленник.

Остановился „трам”: мы - выходим.

\section*{37.}

В этой окраине всё швейцарское смыто; домики, воздух, парки, улыбки, — иное всё: мелькают эльзасски с огромными, чёрными бантами; те зарейнские горы — Эльзас; эти — горы Шварцвальда; Юра — отступает; небольшая долина отделяет границу от пригорода; между тем, серолиловые, стильные здания старых эльзасских домов кричат иным бытом, иною, более широкою культурою — несомненным вкусом.

Вот — уличка перерезалась надвое: точно там на мгновение опустился шлагбаум — на мгновение перервать людоход; и потом — приподняться; но деревянная загородка строга.

То — граница.

Здесь швейцарский посёлочек, Риэн; там — баденский Лоррах, где производят осмотр при\marginpar{69}текающим из Швейцарии ландштурмистам; два швейцарских солдата строжайше блистают штыками (они — из романской Швейцарии); на итальянской границе — дежурят солдаты немецкой Швейцарии.

Вот какая-то кучка, покинувши деревянный немецкий барак (где производят осмотр), хлопотливо бежит через уличку; и военный, выйдя из будки, которая рядом с нами, уже проверяет бумаги; другой — смотрим на нас; спутник мой к нему обращается:

— „Странно ведь: перейти вот ведь уличку; и — попались…”

— „?”

— „Мы — русские.”

Улыбается черноусый солдат; и — говорит по французски, кивая на ландштурмистов.

— „Они — голодают.”

Мы поглядываем на кучечку немецких солдат; те — на нас; это — преклонного возраста люди в серожёлтых мундирах; у одного — серебряная голова; он согбенный годами.

И — отчего-то неловко: отходим мы прочь.

\section*{38.}

День лазурен, прохладен: на веранде тихого пансиона сидим и пьём кофе; спутник мой,\marginpar{70} доктор гейдельбергского университета (он — русский) вспоминает годы студенчества: —

— умер вот Вандельбанд; Риккерт профессорствует теперь в Гейдельберге; убит Ласк на войне; у Гуссерля убит сын… —

— Под ногами, внизу, за холмом, саженях в тридцати от нас, на лужайке треплется маленький, почти игрушечный флаг; то — граница; вечером замечтаешься, голову кверху поднимаешь (неосыпное небо дрожит: переливается звёздами!); и — мечтатель, в Германии ты; тебя подстрелят, наверное… Сторожевые немецкие будки явственны там на склоне горы. Тихо реет военный баллон в нежном воздухе над Эльзасом; и какая-то дама на него поднимает лорнетик.

В память врезалась мне эта тихая, мирная местность: крутобокие горы пушились набухшими почками; и оснежались долины шапками зацветающих вишен; по Германии бежал поездок (миниатюрный какой-то); ландштурмисты миролюбиво поглядывали; и — высилась тихая вилла с полного холмика Бадена.

Через несколько дней я узнал: кровавое происшествие было здесь; не говорили о нём; прошло оно неизвестно: русские пленные, убежав, переходили границу; трое были убиты; чет\marginpar{71}вёртый — попался. Говорят, эта тихая местность кишит шпионажем; и, попадая в глухую деревню, вы видите: подозрительный взгляд у окна, провожающий вас.

Грозовая туча войны здесь повсюду вьедается в воздух.

\section*{39.}

Порою мне кажется: запахнувшись в свой плащ, на глаза придвинувши шляпу, кто-то сядет на „трам” у немецкой границы; прикрывая впадины черепа, будет бегать по Базелю; и затреплет детей костяную рукою своею; или — сядет в таверне, — „инкогнито”: над газетным листом хохотать разорвавшейся челюстью.

Так мне кажется.

Есть в базельской галлерее гравюры Гольбейна; серия их называется: „Danse macabre”; жизнь королей, поселян, духовенства проходит в ней; но скелет сопутствует этой жизни; он так лукаво вплетается и — плутовато подмигивает…

Я гуляю по кривеньким улицам; и безотчётно мне кажется: там, у серого бока домишки, из-за лиловых глицаний — просунется череп; и плутовато оскалится на усталую кучку людей, собирающихся у бассейна, где свои богомольные руки слагает прелатик… на каменном столбике.\marginpar{73}

По горбатеньким уличкам бегают осенями туманы; и — мокрые глянцы; и рыжими пятнами тускловатые фонари освещают дома; под фонарём, запахнувшись в свой плащ, плутоватое инкогнито, смерть, приподымает там шляпу.

\section*{40.}

„Бездники” — русские, мы: уплощены люди запада; тело запада, роковое — „оно”, почиющее в гробе: с лёгких слов Достоевского, почти с легкомысленных слов: — из драгоценной гробницы соборов, как жёлтая мумия, по убеждению нашему, в утро сознания нашего Запад вперялся, — вернее, в зелёное-раззелёное петербургское утро, в котором мы спали: сознание петербуржца цветёт ведь в полуночи.

Но как знать: может быть слушало и „оно” тело Запада, как с горизонтов сознания медленно уплотнялась гроза, громыхающая у меня за окошком уже девятнадцатый месяц — без передышки, без умолку?

Непрерывно гремит кончик фронта; непрерывно гремит за ним фронт; непрерывно гремят все четыреста километров, быть может.

Мне отчётливо ведомо, что все новые сотни тысяч людей, точно рожь в молотилки, ввергаются в те же полосу: отгреметь у орудий; и,\marginpar{73} отгремев, может быть, опочить; от машины — к машинам — идут себе люди; \emph{в гремящую полосу} (здесь на фабриках много есть иностранцев); гремящая полоса — острие всей культуры; вырвалась из руки человека — машина: сроились машины; и — бьют человека: существа непонятных, уродливых, многовиднейших форм — существа грозных демонов! — нашли себе тело; в железе и стали. И обстали дома: безобразными грудами.

\section*{41.}

Это всё приходит на ум при посещении иных из курортов Швейцарии, ныне пустующих.

Они — мёртвые города.

Многорядица друг на друге сидящих, друг друга давящих отелей — ужасна; и ужасен отель, взятый порознь; бестолковый, огромный, чудовищный, каменный куб: городская жизнь — безобразный кубизм; да, мы все — провалились в кубизм; направление в живописи не при чём: направление нашей собственной, мёртвой жизни оно отражает удачно.

\section*{42.}

В брошенных городах раздаётся ужасно: кубизм нашей жизни, снаружи прикрытый цве\marginpar{74}тами, как вот этот чудовищный куб („Palace” иль „Splendide” — всё равно), перед которым для вида разбили пестрейшую клумбу и глупо воткнули две пыльных пальмы, чтобы они веселили взор пёстрой жизнью растительности; всё равно: безобразный куб разве спрячешь? Из него через две пыльных пальмы на вас прёт оторопь \emph{„непокойного дома”}, пустого, где ещё резонируют стены — разговорами, думами и поступками обитавших здесь фатов и модниц пяти частей света; слов не слышите вы; и поступков не видите; но внимаете жестам жизни, здесь бившим недавно, ещё до войны; события — пролетели; а жесты — остались; они ютятся в обоях, в коврах, в плюше кресел: и поднимаются пылью теперь; и покрывают столбами вас этой (глазу невидной, душе же отчётливой) пыли: и пыль — ужасает.

Непосредственное впечатленье предметов носит долго печать обладателей; а впечатленье домов сохраняет печать обитателей; обитатели и посетители дома меняют самое впечатление стен; безобразие самих стен благообразится благообразием жизни; обратно…

Вот — кафе, пансионы, отели, курзалы и клубы: пустые, большие, тяжёлые, каменные; и — сумасшедше-тупые: дико смотрит бессмыслица окон;\marginpar{75} торчат рои труб; безобразно оскален подъезд; стоит хохот подъездов.

Пустая действительность камня пред вами изобличает пустую действительность здесь отхлынувшей жизни; её обнажённый костяк — вот, пред вами: чудовищный каменный \emph{куб} c… квадратами окон; и — две пыльных пальмы.

И эта жизнь есть \emph{„Splendide”}…

Здесь, по каменным тротуарам, под пеклом, утирая усиленно пот, волочились с цветками в петлицах ленивые снобы всех стран в белоснежных суконных штанах и в кургузых визитках; здесь они флиртовали, отплясывая „танго” всех стран: изо дня в день и из месяца в месяц; всё так же, всё те же — дамы в газовых платьях, полуоголённые, напоминающие стрекоз, здесь стреляли глазами в расслабленных \emph{„белоштанников”}…

Теперь — всё не то.

Пусты — рестораны, курзалы, отели: смешной \emph{„белоштанник”} — ненужный, надутый — протащится, дёргаясь, из хохочущей пасти подъезда — куда-то; он не знает — куда: остановился; и — смотрит он, как стоит полисмен, как протащится \emph{трам} (совершенно пустой), как пройдёт полногрудая дама с огромнейшим током на шляпе — в кричаще зелёном во всём;\marginpar{76} из под сквозной короткой юбчёнки дрожат её икры; и до ужаса страшен её смехотворный наряд, заставляющий ждать, что она вдруг припустится в танец; но глаза её — грустны и строги; и — как бы говорят: — „ну за что меня нарядили во всё это”…

Её жалко… до боли…

Может быть: её муж залегает в траншеях; может быть, — в эту минуту бросается он в рой гранат; глаза — плачут; и — там они; а посадка фигуры, походка и \emph{„всё прочее”} моды заставляет несчастную модницу продолжать \emph{„danse macabre”} в каменных тротуарах умершего города.

Дама — в испуге: а „белоштанник” — бодрится; и развинченной, дробной походкой бежит ей навстречу. Вот уже он в кафе: и ему, одному, неизменный оркестрик венгерцев вижжит что-то скрипками.

Но забвения — нет. Нам поэт говорит, будто

\begin{Verbatim}
В бездне бесцельности —
Цельность забвения.
                                                          \emph{Бальмонт.}
\end{Verbatim}

\emph{„Бездна бесцельности”} — сотни и тысячи \emph{„белоштанников”}, сотни и тысячи стрекозящих франтих, заполнявших недавно\marginpar{77} здесь всё; эта „бездна” отхлынула; \emph{„цельность забвения”} — \emph{марево}: действительность повела нас от стен этих зал \emph{через фронт} к поискам живой жизни: не этой.

Оттого-то ужасны здесь отложения жизни — теперь, оттого-то ужасны пустые кафе и отели; многоглазые чудища расхохотались; и — дразнят: подъездами. Некому подъезжать; но и — не к \emph{чему} подъезжать.

Прекрасны слова из мистерии Штейнера: „У врат посвящения”, живописующие один из моментов самопознания человека.

\begin{Verbatim}
И вот!… Теперь, —
Воистину в моих глубинах трепет!…
Вокруг маячит мгла;
Во мне зияет сумрак, взывая мглой миров, звуча из бездн души:
„О человек, познай себя!”
(Из ручьёв и из скал раздаётся
„О человек, — познай себя!”)
Меня меняет сумрак;
Меня меняет бег дневных часов.
В ночи блуждаю я.
И следую в мирах за орбитой земли.
В громах — раскатываюсь;
И мерцаю — в молньях.
Я есмь!… Погаснувшим
Я чувствую в себе себя.
И вижу собственное тело,
Как существо чужое, — вне себя,
И от себя далеко…
Познание дало мне силы
Перенести себя в другом.
\end{Verbatim}
\marginpar{78}

И далее:
\begin{Verbatim}
На изжитую жизнь меня
Ты поворачиваешь снова.
И — как мне вновь познать себя?
Лик человека я утратил:
Мне дикий червь мерещится
В усладах страстных вставший, —
И ясно ощущаю,
Как мглистый образ морока
Чудовищный мой лик
До времени в своих глубинах скрыл.
Моих глубин меня поглотят бездны…
\end{Verbatim}

Момент самопознания человека переживает теперь человечество в целом. Самопознание — горестно; то, что таинственно жило под кровом дневного сознания в нас, — разлилось вокруг нас; в \emph{громах раскатывается и мерцает в молньях}. Форма же прошлой жизни, это — тело культуры, стрясённое грянувшим кризисом — как \emph{существо чужое}: мы теперь созерцаем его таковым, каким было воистину это тело, загримированное утонченнейшей модницей; грим отстал, смыт войной; и мы — видим (быть может каждый из нас), —\marginpar{79}

\begin{Verbatim}
Как мглистый образ морока
Чудовищный мой лик
До времени в своих глубинах скрыл.
\end{Verbatim}

Мы должны теперь, обвиняя других, обвинять и себя; и созерцая чудовище браней, грамящих повсюду, сказать им:

„Да, я — это ты!”

\begin{Verbatim}
Познание дало мне силы
Перенести себя в другом.
\end{Verbatim}

Иначе:

\begin{Verbatim}
Моих глубин меня поглотят бездны.
\end{Verbatim}

Рудольф Штейнер в \emph{„Пути самопознания человека”} великолепно рисует то страшное состояние сознания, которое подстерегает нас на грани двух состояний сознания: „Чувствуешь себя как бы окружённым грозою и бурею. Слышишь гром и видишь молнии. Чувствуешь себя пронизанным силою, о которой дотоле ничего не знал. Потом чудится, что видишь в стенах вокруг трещины. Хочется сказать себе самому…: дело плохо; молния ударила в дом, она настигает меня; я чувствую себя схваченным ею; она меня уничтожает”\footnote{„Путь Самопознания Человека” стр. 19.}.

Не чувствует ли себя человечество ныне пронизанным страшною силою? И „дома” наши нам не дали ли трещины? Дело плохо; молния в нас ударила; уничтожает она.\marginpar{80}

\section*{43.}

Осенью, во второй год войны, я приехал в Монтре; и — бежал.

Монтре — мёртвый город.

\emph{„Белоштанника”} видел я; он напомнил мне дикаря, анахорета мёртвого города, распевающего печально о прошлом.

Среди пения птиц и ручьёв я смотрел себе под ноги, где у озера омертвеневшим пятном расползалася безобразная бугорчатка из каменных, маленьких кубиков, как растущий лишай на цветущей природе.

Таков Монтре с гор.

Изредка я опускался, теряясь в объятиях зданий; пересекал их пустые рои; тротуары, лестницы, крыши казались вселенной, а вселенная гор за Монтре из Монтре принимала вид обыденнейшей цветной фотографии (прикосновения к пошлости, всё опошляет).

Мне казалося: эти кубы домов, безобразно огромных, бесстильных, разгромоздились над бездной; их удел — оборваться в \emph{ничто} или же: медленно раствориться в бездонном по образу и подобию облак; рассеятся маревом; в \emph{никуда} и в \emph{ничто} поднималися стены домов; и \emph{ничто} глядело из окон; выбивались\marginpar{81} ковры; клубы пыли валили с пустого балкона; с веранды и глупо, и пусто грустнел бедный сноб: в \emph{никуда} и \emph{ничто}.

Разъялись иллюзии будто бы многокрасочной жизни; её краски — татуировка; и бронзовеет под нею дикарское тело; и каменеют под пёстрыми амулетами дамских мод — тела \emph{„каменных баб”}; вспоминаю невольно: ещё недавно приняли многие ницшевскую \emph{„blonde Bestia”} просто \emph{„бестией”}; оказалося: \emph{биологический} блондин, \emph{„blonde Bestia”}, есть — болван: пережиток каменного периода, неизвестно как попавшего в будущее, нам оттуда грозит: омертвением, одичанием жизни.

Праздность жизни — дикарство.

Дикари — декаденты; они — обломки культур; неосмысленность утончения жизни — разъедает культуру; и низводит к дикарству; утончённость экзотики, стилизации и искуственный примитив — переходные стадии от культуры к дикарству; и футурист (Парижанин, Берлинец, Москвич — всё равно!) — переход к дикарю.

Карфагенские бритвы в позднейшем периоде жизни встречаются: у танганайского негра; там они — боевые ножи; так всегда: футуристические манифесты о разгроме искусства обернутся действительностью; томагавок \emph{„грядущего хама”} грозит Джиоконде.\marginpar{82}

Среди нас, в городах, образуются новые племена: папуасов XX века; в многообразии проявлений бежит папуас среди нас; он — \emph{„тангист”}; он \emph{„апаш”}; \emph{„футурист”} есть одно проявление; \emph{„белоштанник”} — другое.

В настоящее время с нас сдёрнуты: украшения, амулеты и кольца; лики мертвенной жизни восстали: кричат; безобразие мёртвых курортов — кричащее проявленье дикарства XX века; пока била в них жизнь — мы пьянились её кричащими блесками; но эти блески суть перья и кольца, которыми нас обманывал \emph{папуас}, утверждая, что он — европеец; и мы — ему верили; и танцевали мы — кек-уок, негрский танец; и \emph{„кек-уоком”} пошли мы по жизни; и \emph{„кек-уоковой”} поступью бродит доселе один — грустный фат; в мёртвом гододе: печать \emph{„Кек-Уока”} и \emph{„Танго”} — отпечатлелися на всём проявлении — в нашей жизни; и она — печать дикаря, которого якобы цивилизацией рассосала Европа; не рассосала — всосала: его огромное тело в своё миниатюрное тельце. И Полинезия, Африка, Азия протекли в её кровь: в ней вскипели; в ней бродят и бредят: уродливо-дикой фантазией, беспутницей плясовой изукрашенной жизни: бытом, стилем и модами; и даже — манерой держаться.

Европа — мулатка.\marginpar{83}

Что-то дикое есть в безобразии стиля домов, в сумасшедшем взгляде пустых мрачных окон отелей, в глухих звуках \emph{гонга}, призывающего в час обеда к огромным столам… \emph{„Никого”}.

Мёртвый город — курорт — без людей напоминает ряды огромнейших черепов, оскаленных подъездаными ртами; это — смерть; и от неё мы должны отрешиться: создать город жизни — \emph{„Град Новый”}: Град Солнца.

Если мы не осознаем ближайшей задачи своей, то мулатский облик Европы из шоколодно—лимонного станет… бронзово-чёрным; и из лёгкой личины «утончённой» кек-уоковской жизни вдруг оскалится морда негра: томагавок взмахнётся.

Негр уже среди нас: будем твёрдо… арийцами.

\section*{44.}

Говорит мне знакомый: „Вы поедете на Дуриго?… Дуриго чудесно поёт…”

— „Посмотрите: а звёзды-то… звёзды?”

— „Там пролетели на днях аэропланы”.

— „Ах, то-то стреляли…”

В лиловой багряности те же тучи несутся; и — тот же Юпитер с Венерой: с любовию — мудрость. Соединение в небесах, а на земле — разделение.\marginpar{84}

— „Ну и так — на Дуриго?”

Но о Дуриго не хочется думать; пусть все едут послушать Дуриго. Не поеду я на Дуриго; не надо Дуриго. Все хотят поразвлечься. Развлечения в Дорнахе редки; буду же развлекаться и я; у меня развлечение есть: Александрийский период культуры, о котором я думаю.

\section*{45.}

Восток или запад?

Вопрос — „огненный”: не потому, что в нём слова „восток”, „запад”, а потому что этими словами мы неожиданно выдали нашу страшную тайну, что всё — умерло, провалилось и сгнило в нас; так что мы уже черпаем силу во вне (не в себе): на востоке, на севере, северо-востоке и юге…; основные восприятия нами культур, быта, мыслей, космическим сдвигом выброшены из нас во вне: на восток и на запад; в таком случае наш вопрос — подбиранье частей нашего сердца, вырванных из груди и раздавленных народами; и сказать: „я — восточник” это значит сказать, ну — например: \emph{я — без носа: у меня он был, но он… Я нашёл его на востоке: великолепный нос, из слоновой кости — попробуйте…}\marginpar{85}

С постановкою этого рокового вопроса выдаётся признание, что привычки, быт, моды, искусства, культуры и мысли суть трупы, которые заражают нам воздух, и которые мы должны бальзамировать во избежание всеобщего мора и отнести в музей — к мумиям; наши „востоки” и „запады” — мумии нашего духа; огненно признание это; огненна наша боль, что не \emph{люди мы, а — западно-восточные} трупы; ощущение страшного громового удара сопровождает наш вопрос — молнию: „восток или запад”?

Обратно.

Вопрос — „молния вопрос” не потому, что интимнейшие биения духа в нас мертвы, а потому что биения этого духа нам разрывают границы пространства и времени, что в человечестве вспыхивает пожар: пожар жизни духа; перегорает бывшая черта между \emph{„вне нас”} и \emph{„в нас”}, так что всё вне-лежащее, отложенное и умершее некогда… воскресает; что подобно Тихо де Браге, Копернику, Кеплеру, разорвавшим тесное небо в безбрежность, мы рвём ныне время с историей (его плотью); что когда-то бывшие на западе и востоке культуры повосставали из смерти — бросились в душу: быть интимнейшей составною частью души и её разрывать в мыслях, стилях, вкусах, стремлениях, чаяньях, что не мы\marginpar{86} стали мумией, а мумия фараона Рамзеса II к нам вышла из своего стеклянного гроба, что история — кончила быть, и что времени — нет.

Огненно признание это; огненна наша тайная радость, что не только воистину воскрес к жизни Христос, но что и мы в нём воскресли.

Но вернее всего, что два полюса (жизнь и смерть) одновременно скрестились в вопросе восток или запад? И умерло, разложилось и вывалилось из души (на \emph{восток} и на \emph{запад}) её историческое, преемственное представление о содержании всего: мысли, быта, культуры, истории, устремлений и чаяний от какого-то глубинного удара души, от которого у современного человека надвое была расчленена грудь и был вырван язык; вот он — труп; не оттого ли: что так обострено сознание лежащего трупа (мировая война показала, что \emph{трупы} не умерли), не оттого ли его должны осенить и надежды в его трупном лежании, что он встанет и скажет:

\begin{Verbatim}
Как труп, в пустыне я лежал…
И Бога глас ко мне воззвал:
Восстань, пророк, и виждь и внемли,
Исполнись волею моей,
И, обходя моря и земли,
Глаголом жги сердца людей.
\end{Verbatim}

Пронесётся тогда, что —\marginpar{87}

\begin{Verbatim}
Открылось! Весть весенняя!… Удар молниеносный!…
Разорванный, пылающий, блистающий покров!
В грядущие, громовые, блистающие весны,
Как в радуги прозрачные, спускается… Христос.
И голос поднимается из огненного облака:
„Вот тайна благодатная, исполненная дней”!
И огненные голуби из огненного облака
Раскидывают светочи, как два крыла, над ней.
\end{Verbatim}

Покров — душа наша: огонь духа сожжёт её; если в ней духа нет, то сожжённая, она явит нам труп: наше тело; если искра духа в ней есть, то он — будет светом нашей телесной лампады, где душа — только масло, которого назначенье: сгореть.

Восток или запад? В этом вопросе — первое дыхание бурь огня, от которого уже пылает земля и который сожжёт в нас до тла — всё что в нас не огонь.

Восток или запад?

\section*{46.}

Одинаково остр вопрос — на востоке и западе: в России, в Китае, в Европе, в Америке, на Сандвичевых островах, у бурят…

Кто \emph{„мы”} — чукчи, бурята, немцы, русские, малороссы, литвины, иль… люди? И кем должны стать: обитателями провинции, страны, континента,\marginpar{88} или же — обитателями вселенной, участниками космической жизни, равноправными гражданами всех планет и всех солнц?

И говорят нам: \emph{„Мы — запад”}. Но на западе \emph{„запада”} нет. И говорят: \emph{„Мы — восток”}. Где „восток”?

Что „востока” и „запада” нет, было ведомо Гете и очень многим до Гете; почему же всё-таки:

— „Восток или запад”?

Да потому что всё — треснуло: омертвенела культура; и — валится; что какой глубинный росток, пробиваясь наружу, рвёт её умирающие и набухшие части; бывшее центром жизни вытолкнуто к периферии, во вне; и мы ходим чреватыми.

Столкновение двух культур, душ и рас раскололо нам душу — о дух; расколотые половинки души симметрически закачались и выпали (из нас — во вне нас) „западом” и „востоком”; и что было не видно доселе, стало видно теперь.

Охватить грандиозности кризиса нам нельзя в потрясении нашем; все охваты умеренны; все \emph{„востоки”} и \emph{„запады”} лишь пристойная маска мирового скандала; произошла огромная непристойность: жизнь треснула; но по правилам доброго старого времени мы стараемся всё ещё игнорировать трещину, а расколы жизни смягчить: затупить в антиномию меж германцем и русским,\marginpar{89} мужичком и чиновником, славянофилом и западником.

\section*{47.}

Не понимаю я деления на „восток” и на „запад”: передо мной серия многообразных делений по национальностям, по идеям, по вкусам и по периодам времени; эти серии \emph{„западов”} и \emph{„востоков”} напоминают материю цвета \emph{„шанжан”}; скажешь: „Вот, вот — восток”. Отойдёшь на шаг, скажешь: „запад”; отойдёшь на четыре шага, и — „восток”.

— \emph{„Азия была народовержущим вулканом”} — так когда-то напыщенно сказал с кафедры один профессор истории \emph{Гоголь-Яновский}. Для известного периода — да; вообще — нет и нет! Северно-европейское происхожденье „востока” есть факт науки; \emph{„народовержущий вулкан”} передвинут: он — двигался: с севера Европы к востоку и югу; но — опять-таки: как попали в Испанию древние изображения ацтеков? И откуда запали в архипелагские древности пернатые краснокожие? Геология даёт право нам думать об исчезнувшем континенте меж Америкой и Европой, о мифической Атлантиде; в ней — начало \emph{„западов”} и \emph{„востоков”}: \emph{„народовержущий вулкан”} тут, и из неё\marginpar{90} струя лавы через Европу на Азию — оплодотворяет Азию и в сумеро-акадийской культуре слагается в тот „восток” собственно, который не так уже древен и не вовсе восточен: история и деликатней, и тоньше провозглашенья о ней профессора истории Гоголя в громовержущей реторике слов:

— „Азия была…” и т. д.

Гоголю простителен этот образ; но он нам непростителен; непростительно деление в наши дни на деторождающую, безмозглую Азию и на Европу — бездетную, но… с идеями: на \emph{„восток”}, и на \emph{„запад”}.

Таким „западом” окажется Герберт Спенсер для нашего русского западника (до и после — „востоки”); и таким „востоком” окажется наша Русь, если мы её сложим по образу и подобию старинного быта: „извержение чад” будет в ней; извержение мыслей — не знаю.

\section*{48.}

Наши „западники” и „восточники” населяют умственный мир несуществующими „западами” и „востоками”.

Заратустра — восток или запад? Географически он восток, а, по правде сказать, он — конечно же — „запад”: его связь с Гераклитом\marginpar{91} и далее с Гельдерлином, Новалисом, Демелем, Моргенштерном (великолепнейшим современным поэтом, недавно скончавшимся), с Ницше — установима отчётлива; стоит взять в руки „Гаты” Ясны, — т. е. гимны из книги, приписанной Заратустре: солнечная гуманная ясность и утверждение личности — в ней; Заратустра — солнечный щит, защитивший некогда запад от злого мрака туманства; он — „западник”; и, конечно, Кант учредитель Китая: во всём строе мысли Шопенгаэур — „индус”, провозглащающий незыблемость истин Веданты и обращающий провозглашение это в эпиграф системы; но рождаются в нём — Ницше, Вагнер: и ими пульсирует запад.

Что такое Фриц Маутнер, Вильгельм Вундт, Бэнно-Эрдман и очень многие прочие? Двое первых пригвождают мысль к корню слова: а корень слова — „нутро”; выводят из безглагольного и чисто восточного взвизга к физиологическому восточному взвизгу „нутра”; солнечность смысла слова и мысли стираются ими. А Бэнно-Эрдман, психолог и логик, гласящий, что суждение \emph{„являет собою течение словесных представлений, которому не соответствует никакого значения”} — он „восток” или „запад”? Утверждениями этими разве не явно оскалился на нас западный\marginpar{92} Сфинкс эфиопский своей гримасой; в утверждениях этих с запада прёт \emph{„восток”}.

Борьба „западов” и „востоков” — борьба химеры с драконом; оба — мифы, не уплотнённые ещё никогда и желающие воплотиться впервые; воплощение мифов есть выход: из замкнутой исторической жизни в незамкнутость жизни мифической; и из неё — в дали космоса: на рубеже её, угрожая и застилая нам путь; перед нами встают „восток”, „запад”: западный пролетарий духа и Ксеркс; восточный парий и… духом играющий ницшевский Заратустра меняют обличия; меж ними — едва заметный пролив в океан новой эры: там ждёт нас Видение Будущего — не восток и не запад:

\begin{Verbatim}
О Русь! В предвиденьи высоком
Ты мыслью гордой занята:
Каким же хочешь быть востоком —
Востоком Ксеркса иль Христа?
                                 \emph{Вл. Соловьёв.}
\end{Verbatim}

\end{document}
