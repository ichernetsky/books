\documentclass[12pt,a4paper,oneside]{book}
\usepackage{fontspec}
\setmainfont[Mapping=tex-text]{Linux Libertine O}
\setmonofont[Scale=0.8,Mapping=tex-text]{Linux Libertine O}

\usepackage{polyglossia}
\setdefaultlanguage[spelling=modern]{russian}
\setotherlanguage{english}

\usepackage{indentfirst}
\usepackage{fancyvrb}
\fvset{xleftmargin=4cm}

\newenvironment{poem}{\begin{Verbatim}[xleftmargin=4cm]}{\end{Verbatim}}

\setcounter{tocdepth}{0}
\setcounter{secnumdepth}{1}

\begin{document}
\title{На перевале}
\author{Андрей Белый}
\date{}
\maketitle

\tableofcontents

\part{Кризис жизни}

\subsection{Вместо предисловия}
Предлагаемый „дневник” мыслей есть часть дневника, который пришлось мне вести в Швейцарии в 1915-ом и 1916 году; части из этого дневника в своё время были мной напечатаны в отрывках; другие же части вошли в мою книгу „Кризис сознания”, увы, не могущую появиться на свет по условиям нашего времени. Перечитывая этот дневник, убеждаюсь невольно: не устарел он; охвачены тем же мы легкомыслием; события, ударявшие нас, озлобляли нас друг против друга; на себя самих не повернулись доселе мы.

„О человек, познай себя!”

\begin{flushright}Андрей Белый.\end{flushright}

Москва. 1918 года, июль.\marginpar{5}

\section*{1.}

„Гремящая тишина”!

Девятнадцатый месяц со мною она в мертвом шелесте городов, в мертвом беге часов; утром, ночью и днём — все гремит с горизонта.

Есть особая тишина у швейцарской провинции этого угла Базельланда, в котором засел я давно; неподобна она тишине русских ширей, где сердце бунтует, где все — необъятно, где ветром несутся пространства; и падает небо на вас самоцветною звёздочкой; всё под ним отступило: всё — плоско; всё — ровно; глаза упираются в переливы заката и в кудри косматого облака; и тоска или радость, от который нет выхода, угоняют вас прямо в смерть.

Здесь, в Базельланде, всё — скучно; всё — скученно: несуетливо, но — душно; и внятно гласящее небо здесь часто закрыто, и внятно светящее слово\marginpar{7} зажато в гортани неповоротливых обитателей двух деревушек, между которыми поселился я до войны; жители Арнсгейма и Дорнаха не внимают давно уже голосу внятно гласящих орудий с эльзасской границы.

Битвы в Эльзасе обычны: как падение местного водопадика, эти битвы сопутствуют вашей жизни; вы их слышите: говором пушек — оттуда, с границы: вот „оно” — загремело: гремит.

И гремело так год назад; через год — отгремит ли\footnote{С той поры, как написаны эти строчки, прошло уже более двух лет; уже около двух лет я в России; а вокруг — всё „гремит”. \emph{Прим. автора}}?

Обыватели местных посёлочков собираются посмотреть на фонарь, только что поставленный меж двумя деревушками, как ходили когда-то под праздник они любоваться с холма на чуть видные огонёчки шрапнелей — \emph{оттуда}.

\section*{2.}

Нас обстал кризис жизни: на перевале сознания подстерегают нас кризисы жизни; приложенья к техническим производствам культуры плотнят нашу мысль: не живая, она превратилась в абстракцию; материальное тело абстракции — машина.\marginpar{8}

Машина восстала на нас: мир стал — мир материально-машинный: и чёрствый, и чувственный; чёрствая чувственность — роковой наш удел.

Мир природы — преставился: ненормальная вытяжка из природы его заменила; утерялась в нас „вещность“, сменяясь экстрактом; и нет нам предметов, а есть предмет-\emph{ины}.

Чревоугодие материальной культуры — продукт очерствления.

Слепота — тончит ухо, а глухота — тончит глаз: неужели же для утончения зрения мы долждны протыкать барабанную перепонку? Это было б безумием. Но на безумии этом построен рост знаний; богатство машинного мира разростаются в мире ценой оглушения, иль ценой ослепления; глухие, слепые, немые вершат нашу участь.

\emph{До столького} дожили мы! До чего доживём, я не знаю.

Мы не видели удалённых молний грозы; мы увидели зарева сожигаемых зданий; расслышали — пушки; лёгкий говор сознания и голоса сознающих ещё — всё ещё! — не расшибли, на нас глухоты; не расшибут они — в будущем.

Голоса наростающих громов культуры — гремели столетия…

Если б нам уши!\marginpar{9}

\section*{3.}

Лучшие традиции Возрождения мы столетия низводили на нет: убивали столетья конкретную значимость жизни; и — говор явлений; Возрождение призывало нас явственно: полюбить все явления мира; и в Возрождении по отношенью к явлениям жизни художник с учёным сливается; художник глядит на явление — мудро; учёный явление греет, ласкающим опытом.

Таков Леонардо: наука его красотою пронизана вся; а искусство в нём — мудро; любовные опыты — опыты Леонардо-да-Винчи.

Но опыт Роджера Бэкона из средних веков — уже пытка явленья: убийство явленья; терзанье, кромсанье его; раскромсанье предметов, убийство предметов перенесли мы в XVI век вопреки всем вершинным традициям гуманизма; часто опыт природы был пыткой ея; так: XVI век, зацветя инквизицией и закруглившись в барокко (в развратно-утончённом Style jesuite), перенёс инквизиционные приёмы терзанья, пытанья в мир целой, цветущей природы; в опытах разрушались предметы для добывания всевозможных гастрических лакомств: и вещь и реальность, как цельное нечто, распались от этого: на абстрацию (пресловутую „вещь в себе”) и\marginpar{10} на труп от конкретной реальности, на феномен, на „вещь для нас”: на продукт потребления буржуазной культуры; материальное тело культуры её превратило в… часть брюха: в отложение жировых желез, в субъективистическую отрыжку действительности; развитие философии сосредоточилось на методологической разработке всевозможных \emph{отрыжек}; и пошли рости „научные” феноменализмы и скептицизмы.

Style jesuite, развитие материальной культуры, номинализм новейшей формации философии коренятся в едином источнике: в разложении конкретного мира на абстракцию и на вытяжку для гастрических потреблений; но в гастрическом потреблении — ещё полной реальности нет, и вкусовая отрыжка комфорта — не действительность вовсе; точно так же: в теоретических выводах специальных отраслей знания перед нами не мир, а разве что… проэкционный пунктирик: да, понятия именно в наших точнейших науках сведены часто к графике; и объяснить, понять — это значит: изобразить сеть кривых и условно исчислить их; дифференцировать — еще не значить: учить пониманию; и чертить графы — не значит осмысливать.

Так первичная конкретность идеи о конкретном предмете подменяется в нас эмблемой.\marginpar{11} Эмблемами мы исчислили необходимость войны; эмблематически прикинули военные партии всего мира размеры добычи; проэкционным пунктириком изобразили учёные инженеры возможные орудия истребления; возникали науки об уничтожении себе подобных; не забуду я никогда: еще будучи гимназистом, я нашёл на столе у отца два почтеннейших кирпича, испещрённых внутри крючковатыми знаками интегралов и функций; это было два руководства; одно называлось: „О внешней баллистике” (о движении ядра вне пушечного жерла); другое же называлось: „О баллистике внутренней”. Две почтенных науки об уничтожении себе подобных блистательно развивались; и бескорыстное открытие Лейбница (дифференциальное исчисление) применили таки мы к войне; преподавание метода убивать своих ближних разработали математики, инженеры, механики, техники культурнейших, цивилизованных стран; сотни тысяч убитых убиты еще до рождения: быть убитыми предначертаны.

И знай Лейбниц, что в лучшем из миров открытие его ляжет в грядущее массовым истреблением людей, колоссальнейшей бойнею мира, — как знать: может быть, своё открытие сжёг бы он.

Мы браним нынче Круппа. Нашёлся обще\marginpar{12}ственный деятель, соединивший с Круппом… и философа Канта. Но… почему Канта именно?.. Надо брать — раньше: Лейбниц — виновник теперешней бойни народов; или вернее: за Лейбницем спрятанный, тонкий гастроном культуры, вооруженный наукою, как ножом, для… мирового разбоя. Появился же этот разбойник, как прямое наследие отношенья к явлениям жизни: в тот момент, как идея в явлении угасает, явление есть предмет потребления; но явление для меня — предстоящее всякое, „ты”; и оно, это „ты” потребление.

Вивисекционные опыты с жизнью — они породили ту бойню, в которой живём: и не Лейбниц, а ранее Лейбница появившийся Бэкон, быть может, виновник характера современной войны.

Раз идея в явлении пропадает, явление — предмет потребления; и оно начинает тогда округлять нам желудок; \emph{„капиталистическое”} проявление желудочной деятельности разростается в нас; наш желудок теперь вывисает из нас толстым брюхом; и мы — брюхоногие пауки, а не люди; конкретности жизни нам — жир; идеалы живые — пунктир на бумаге, рисующий в знаках законы… баллистики; истина есть „не сущее”;  и оттого-то в „не сущее”\marginpar{13} принимаемся мы превращать вечно сущие жизни; истребляем и рвём их вокруг.

Вместо слияния с миром — господствует: пожирание мира и раздробление мира; то есть: введение мира в желудок для накопления… жировых отложений. Человек XX века — безмясый скелет, опухающий жиром; вместо \emph{знания} и \emph{сердечного} отношения к жизни у него господствует два усвоения жизни: при помощи мозга и при помощи функций желудка; первое усвоение — \emph{„крап на ничто”} (т. е. крап электронов над бездной); и при помощи этого \emph{„дифференциального крапа”} слагает он на бумаге чертёжики пушек; усвоение же второе — чревоугодие; лишь оно одно доминирует в нём; малокровная мысль, превращённая в крап электронов, становится техникой чрева, изготовляя ему искусственные, многозубые челюсти крепостей, изборождённые пушками.

\section*{4.}

Моё окошко — в долину; цветущие, белокудрые вишни весною глядят из него; вечерами восходят закаты; в него свистит ветер — всю осень, всю зиму; над вершинами низкорослых деревьев — отчетливая черепица домов; дальше — дали, бегущие в линии голубоватых холмов; в\marginpar{14} голубоватом тумане — граница; будто бы иногда распахнётся там воздух: перед ненастьем особенно; и прорежутся темные гребни Эльзаса.

Вот оттуда то и летит:

— „Ру-ру-рууу”…

Порой отзываются стёкла окон; вдруг не выдержат; и — расплачутся; звук немецкой пушки я знаю: отчетливый, надоедливый звук; а вот это невнятное „у-у-у” — вероятно, французская пушка; говорят: из Мюльгаузена и из Бэльфора пушек доносится внятно до Дорнаха.

Так говорят эти пушки — дни, месяцы: девятнадцатый месяц; здесь, в Швейцарии, пушки молчат; но молчание здесь чревато глухим, наростающим взрывом; будут взрывы повсюду; и из груди, как жерла, оторвавшись от жил, точно бомба, взорвётся кровавое, обнажённое сердце;  человек в эти дни, точно пушка: заряжен он кризисом.

Тема кризиса сплетена с возрождением. Тема гибели мира связуема с темой рождения. Не случайны поэтому голоса, нас зовущие к выси духовной: переродиться пора!

Голоса Мережковского, Ибсена, Штирнера, Ницше, Владимира Соловьёва звучали. Звучит голос Штейнера. Выявляя нам нервы культуры,\marginpar{15} гласят очень внятно они о падении великолепных обломков культуры; и — о паденье домов: домов старого строя.

Дома — под обстрелом.

И под обстрелом, быть может, вся эта тишайшая местность: в первый месяц войны, Боже мой, что тут было; появились французы в предместьи Базеля, St. Louis; понадвинулись с севера немцы; и собирались вдавить из Эльзаса в Швейцарию, к нам, передовые французские части; поразвесили объявления о возможности битвы под Базелем; ожидали мы с часу на час здесь сигнала тревоги; по первому знаку сигнала должны были мы налегке пробираться — туда, через горы: черз кряжистый Гемпен, висящий над Дорнахом. По дорогам задвигались швейцарские пехотинцы; трещал здесь и там барабан; батареи уставились по направленью к границе; в пыли забелели султаны; и — фыркали лошади; заскрипели телеги с фуражем; а сумасшедшие, исступлённые кучки кричали, что надо бежать: нейтралитет будет попран. Говорилось тогда об обстреле домов; этот дом — не опасен, а этот — опасно поставлен.

Но опасно поставлен не дом, не окрестность, не даже кантон, не страна; вся культура — опасно поставлена; вся под обстрелом она. Все\marginpar{16} кумиры культуры — в опасности; изображения Вотана, Доннера, Логе — падут; гибель старых божеств волим мы. Старый бог, бог войны (alter Gott): должен пасть!

Рушатся представления о данной действительности; рушатся переживания её; пропадает в нас строй ощущений, будто \emph{„я”} в ней находится; пропадают реальные ощущения \emph{„я”}; действительность убежала от \emph{„я”}; утекла от него; как свинцовая гиря, стремительно погружается в глубину подсознания \emph{„я”}; его целостность точится всекипящим движением мира:

\begin{Verbatim}
В какие-то кипящие колёса
Дума моя, расплавясь, протекла.
\end{Verbatim}

\section*{5.}

Должное, реальное знание — в усвоении предмета узнания; а современное знание, сосредоточившись на методе, предмет упраздняет; предметом узнанья становиться метод; и вне метода — хаось: вращение газов в желудке.

Самые органы чувств порасшатаны современною культурою; и пропитаны — алкоголем, пропитаны — никотином; восприятия органов чувств — \emph{никотинны}; в них луг пахнет дымом; в восприятиях наших природа убита давно; пошлый\marginpar{17} рёв паровоза — неотъемлемая принадлежность обычного европейского пейзажа; и — линия телеграфных столбов; а фабричная гарь — принадлежность зари; естественных восприятий в нас нет; и оттого-то нам нужны абстракции доказательств и самая материя потребления; осязание, грубейшее чувство, оно только живо в нас.

Так предметы узнания уничтожаются нами: в процессе узнания распыляются в голове электронным пунктиром и раздробляются на зубах при введении в „чрево”. Вместо конкретного мира поэтому выростает мир в нас танцующих математических знаков (дифференциалов и функций), роящихся, точно грязные мухи над миром… желудочных отбросов.

Вот — подлинный, неприкрашенный образ материальной культуры; и вот результат: потребления мира природы, её раскромсанья на части; то есть разложение её — нами, в нас, вокруг нас; мир отсутствует в нас и вне нас; мы из мира повыпали; полетели над бездной в… действительность, несоизмеримую с некогда данной от Бога.

Действительность, нам грозящая, прорезается явственно, под покровами умерщвлённой природы; вот она показалась уже, пока видная нам в аппаратах, приборах и лупах, как мир… инфу\marginpar{18}зорий; но аппараты, приборы и лупы воистину суть: наши новые органы чувств; мы испортили наши природные органы; кто-то нам подарил мир искусственных органов, прилипающий к глазам и ушам, приростающий к безорганно висящему малокровному мозгу, протягивающему во все стороны, точно спрут, свои вялые разветвления нервов: высасывать соки природы; на обнажённые нервы, лишенные кожных покровов, насели нечистые мухи, роящиеся над сознанием нашим — математическим знаком; обнажённые нервы естественно бронируем мы: сталью, железом; бронированный сталью, бесформенный, нервный, безжалостный спрут — вот искусственный человек, приготовивший нам мировую войну.

\section*{6.}

Мир бактерий в микроскопах: „Вот я выйду из труб микроскопов; и расселюсь среди вас: бактерии заживут человеками; через трубу микроскопа вы свалитесь все в микроскоп; и — заживёте бактерией”.

Воистину, на земле мы — „как будто”; где её былой лик? Где её конкретная правда? Материальная культура — не культура земли: земля — идеально, конкретна, природна, естественна.\marginpar{19}

\section*{7.}

Люблю землю я: она — горная, кристаллически чистая масса; лишь её поверхностный слой — унавоженный перегной; унавоженный, дурно пахнущий перегной, перепачканный всевозможными отбросами, в представлении большинства производителей материальной культуры — земля. Но земля есть огонь: огонь лавовых струй; и с навозом не смешана.

И земля, — это горы.

Вспоминаю скитанья в горах; мыслишь там — ни о чём; ни о чём — свисты ветра.

Но \emph{никчёмные} мысли летают огромными ритмами; мыслью рушатся горы: в душе водишь думы; идешь себе; уж не смотришь: в полузакрытых глазах метаморфозы обставшего пейзажа сотворяются заново: полуобразом, полумыслью; там линия пиков змеится орнаментом мысли, овеянной ветром; ты — ветряный: в голове твоей ветер; останови его; и — фиксируй: он тотчас же уплотняется силлогизмами; сознанием проницаешь ты ритм вне-сознательной мысли.

Где-нибудь перекусишь; и — далее.

Сознание наблюдает, описывает проростание мыслей из красочных пятен фантазии; и — проростание в эти пятна тебя обстающих громад;\marginpar{20} громады поят тебя мыслью; она — чистая, кристальная мысль, там осевшая горной породой; и здесь вставшая — философемою, как вот эта долина; ты прошёл шесть долин; шестерично переменились рельефы; шесть систем философии пробежало вершинами: Гёте, Гердер, Новалис, Шлегель, Шеллинг и Гегель — прошли пред тобою.

Ты видел во-очию их. Твоя мысль ни о чём, пробуждённая в душу павшими пиками, осозналась; и — вот она: \emph{мысль и природа — одно}.

Человечна и мягко гуманна природа; в ней нет извращений; но человеческий взор озирает её плотоядно; человек современности на неё воззрился, как кошка на птичку; птичье пение — есть; но в зажаренном птичьем мясе нет пения, в „потреблённой” природе идея убита; и — мертвой материей противостоит она нам; материализм — вне природен.

Основа природы — природа идеи; и \emph{философия тождества} Шеллинга проверяема горным ландшафтом; слушай пенье потока; записывай точно, что встанет из пенья потока в душе у тебя; правда Гётевой мысли откроется явственно: \emph{„манифестация тайных знаков”} природы в умении понимать жизнь сознания; метаморфозы идей уподобляемы метаморфозе растительных организмов; метаморфоза идеи — кон\marginpar{21}кретна, точна, наблюдаема, описуема; и описание точной фантазии мысли и есть философия.

Вот гуманнейший лейт-мотив в мировоззрениях Гёте, Гегеля, Шеллинга, Гердера; и — других; и — вещая фантазия мысли (сознание и природа — единство в природе идеи) — гуманна; корни русского самосознания в ней, в этой мысли.

Бросать камнями в эти мысли, не значит ли — откровенно идти на разрыв с нас зовущей природою, эта природа — природа сознания нашего; но подчас преставления о природе в нас подменяются представлением о „трамвайной”, о „материальной” культуре, в которой природа — машина.

\section*{8.}

Всякий знает из нас — вот такую невнятицу: вдруг покажется, что в напряжениях материальной культуры на-двое разрывается жизнь, что машины бьют в пульсы слабеющей жизни железными пульсами; схватит такая минута стремительно в суете городов; и покажется вдруг, что вот он, фешенебельный господин Манекен с угла уличной вывески рекламирует немо толпе производство Гомункула; и — рост механической жизни покажется грозным наростом; пухнет опухоль городов — пора ампутировать опухоль: материальное тело жизни — раздуто чрезмерно.\marginpar{22}

Безотчетности эти переживаем мы все; и горизонты сознания возникают пред нами; горизонты гремят своим кризисом; материальную опухоль жизни пора ампутировать: человечеству угрожает гангрена.

Помню день.

Всё червонилось, багрянело; всё — рдело птичий свист — уносил; говоры оголтелых утёсов гремели, дрожали; рыдали потоки; землевороты бежали под небо и ясногранною вереницей плотнели в тенях, гребни резали небо; серебрились осколки их; прорезались покровы природы, прорезались навстречу природе — природа сознания; и природа в природу сознания поглядела, как Сфинкс; земли не были землями: одухотворились и жили они, как природа идеи.

\end{document}
