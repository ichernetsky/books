\documentclass[12pt,a4paper,oneside]{book}
\usepackage{fontspec}
\setmainfont[Mapping=tex-text]{Linux Libertine O}
\setmonofont[Scale=0.8,Mapping=tex-text]{Linux Libertine O}

\usepackage{polyglossia}
\setdefaultlanguage[spelling=modern]{russian}
\setotherlanguage{english}

\usepackage{indentfirst}
\usepackage{fancyvrb}
\fvset{xleftmargin=4cm}
\fvset{commandchars=\\\{\}}

\setcounter{tocdepth}{0}
\setcounter{secnumdepth}{1}

\begin{document}
\title{На перевале}
\author{Андрей Белый}
\date{}
\maketitle

\tableofcontents

\part{Кризис жизни}

\subsection{Вместо предисловия}
Предлагаемый „дневник” мыслей есть часть дневника, который пришлось мне вести в Швейцарии в 1915-ом и 1916 году; части из этого дневника в своё время были мной напечатаны в отрывках; другие же части вошли в мою книгу „Кризис сознания”, увы, не могущую появиться на свет по условиям нашего времени. Перечитывая этот дневник, убеждаюсь невольно: не устарел он; охвачены тем же мы легкомыслием; события, ударявшие нас, озлобляли нас друг против друга; на себя самих не повернулись доселе мы.

„О человек, познай себя!”

\begin{flushright}Андрей Белый.\end{flushright}

Москва. 1918 года, июль.\marginpar{5}

\section*{1.}

„Гремящая тишина”!

Девятнадцатый месяц со мною она в мертвом шелесте городов, в мертвом беге часов; утром, ночью и днём — все гремит с горизонта.

Есть особая тишина у швейцарской провинции этого угла Базельланда, в котором засел я давно; неподобна она тишине русских ширей, где сердце бунтует, где все — необъятно, где ветром несутся пространства; и падает небо на вас самоцветною звёздочкой; всё под ним отступило: всё — плоско; всё — ровно; глаза упираются в переливы заката и в кудри косматого облака; и тоска или радость, от который нет выхода, угоняют вас прямо в смерть.

Здесь, в Базельланде, всё — скучно; всё — скученно: несуетливо, но — душно; и внятно гласящее небо здесь часто закрыто, и внятно светящее слово\marginpar{7} зажато в гортани неповоротливых обитателей двух деревушек, между которыми поселился я до войны; жители Арнсгейма и Дорнаха не внимают давно уже голосу внятно гласящих орудий с эльзасской границы.

Битвы в Эльзасе обычны: как падение местного водопадика, эти битвы сопутствуют вашей жизни; вы их слышите: говором пушек — оттуда, с границы: вот „оно” — загремело: гремит.

И гремело так год назад; через год — отгремит ли\footnote{С той поры, как написаны эти строчки, прошло уже более двух лет; уже около двух лет я в России; а вокруг — всё „гремит”. \emph{Прим. автора}}?

Обыватели местных посёлочков собираются посмотреть на фонарь, только что поставленный меж двумя деревушками, как ходили когда-то под праздник они любоваться с холма на чуть видные огонёчки шрапнелей — \emph{оттуда}.

\section*{2.}

Нас обстал кризис жизни: на перевале сознания подстерегают нас кризисы жизни; приложенья к техническим производствам культуры плотнят нашу мысль: не живая, она превратилась в абстракцию; материальное тело абстракции — машина.\marginpar{8}

Машина восстала на нас: мир стал — мир материально-машинный: и чёрствый, и чувственный; чёрствая чувственность — роковой наш удел.

Мир природы — преставился: ненормальная вытяжка из природы его заменила; утерялась в нас „вещность“, сменяясь экстрактом; и нет нам предметов, а есть предмет-\emph{ины}.

Чревоугодие материальной культуры — продукт очерствления.

Слепота — тончит ухо, а глухота — тончит глаз: неужели же для утончения зрения мы долждны протыкать барабанную перепонку? Это было б безумием. Но на безумии этом построен рост знаний; богатство машинного мира разростаются в мире ценой оглушения, иль ценой ослепления; глухие, слепые, немые вершат нашу участь.

\emph{До столького} дожили мы! До чего доживём, я не знаю.

Мы не видели удалённых молний грозы; мы увидели зарева сожигаемых зданий; расслышали — пушки; лёгкий говор сознания и голоса сознающих ещё — всё ещё! — не расшибли, на нас глухоты; не расшибут они — в будущем.

Голоса наростающих громов культуры — гремели столетия…

Если б нам уши!\marginpar{9}

\section*{3.}

Лучшие традиции Возрождения мы столетия низводили на нет: убивали столетья конкретную значимость жизни; и — говор явлений; Возрождение призывало нас явственно: полюбить все явления мира; и в Возрождении по отношенью к явлениям жизни художник с учёным сливается; художник глядит на явление — мудро; учёный явление греет, ласкающим опытом.

Таков Леонардо: наука его красотою пронизана вся; а искусство в нём — мудро; любовные опыты — опыты Леонардо-да-Винчи.

Но опыт Роджера Бэкона из средних веков — уже пытка явленья: убийство явленья; терзанье, кромсанье его; раскромсанье предметов, убийство предметов перенесли мы в XVI век вопреки всем вершинным традициям гуманизма; часто опыт природы был пыткой ея; так: XVI век, зацветя инквизицией и закруглившись в барокко (в развратно-утончённом Style jesuite), перенёс инквизиционные приёмы терзанья, пытанья в мир целой, цветущей природы; в опытах разрушались предметы для добывания всевозможных гастрических лакомств: и вещь и реальность, как цельное нечто, распались от этого: на абстрацию (пресловутую „вещь в себе”) и\marginpar{10} на труп от конкретной реальности, на феномен, на „вещь для нас”: на продукт потребления буржуазной культуры; материальное тело культуры её превратило в… часть брюха: в отложение жировых желез, в субъективистическую отрыжку действительности; развитие философии сосредоточилось на методологической разработке всевозможных \emph{отрыжек}; и пошли рости „научные” феноменализмы и скептицизмы.

Style jesuite, развитие материальной культуры, номинализм новейшей формации философии коренятся в едином источнике: в разложении конкретного мира на абстракцию и на вытяжку для гастрических потреблений; но в гастрическом потреблении — ещё полной реальности нет, и вкусовая отрыжка комфорта — не действительность вовсе; точно так же: в теоретических выводах специальных отраслей знания перед нами не мир, а разве что… проэкционный пунктирик: да, понятия именно в наших точнейших науках сведены часто к графике; и объяснить, понять — это значит: изобразить сеть кривых и условно исчислить их; дифференцировать — еще не значить: учить пониманию; и чертить графы — не значит осмысливать.

Так первичная конкретность идеи о конкретном предмете подменяется в нас эмблемой.\marginpar{11} Эмблемами мы исчислили необходимость войны; эмблематически прикинули военные партии всего мира размеры добычи; проэкционным пунктириком изобразили учёные инженеры возможные орудия истребления; возникали науки об уничтожении себе подобных; не забуду я никогда: еще будучи гимназистом, я нашёл на столе у отца два почтеннейших кирпича, испещрённых внутри крючковатыми знаками интегралов и функций; это было два руководства; одно называлось: „О внешней баллистике” (о движении ядра вне пушечного жерла); другое же называлось: „О баллистике внутренней”. Две почтенных науки об уничтожении себе подобных блистательно развивались; и бескорыстное открытие Лейбница (дифференциальное исчисление) применили таки мы к войне; преподавание метода убивать своих ближних разработали математики, инженеры, механики, техники культурнейших, цивилизованных стран; сотни тысяч убитых убиты еще до рождения: быть убитыми предначертаны.

И знай Лейбниц, что в лучшем из миров открытие его ляжет в грядущее массовым истреблением людей, колоссальнейшей бойнею мира, — как знать: может быть, своё открытие сжёг бы он.

Мы браним нынче Круппа. Нашёлся обще\marginpar{12}ственный деятель, соединивший с Круппом… и философа Канта. Но… почему Канта именно?.. Надо брать — раньше: Лейбниц — виновник теперешней бойни народов; или вернее: за Лейбницем спрятанный, тонкий гастроном культуры, вооруженный наукою, как ножом, для… мирового разбоя. Появился же этот разбойник, как прямое наследие отношенья к явлениям жизни: в тот момент, как идея в явлении угасает, явление есть предмет потребления; но явление для меня — предстоящее всякое, „ты”; и оно, это „ты” потребление.

Вивисекционные опыты с жизнью — они породили ту бойню, в которой живём: и не Лейбниц, а ранее Лейбница появившийся Бэкон, быть может, виновник характера современной войны.

Раз идея в явлении пропадает, явление — предмет потребления; и оно начинает тогда округлять нам желудок; \emph{„капиталистическое”} проявление желудочной деятельности разростается в нас; наш желудок теперь вывисает из нас толстым брюхом; и мы — брюхоногие пауки, а не люди; конкретности жизни нам — жир; идеалы живые — пунктир на бумаге, рисующий в знаках законы… баллистики; истина есть „не сущее”;  и оттого-то в „не сущее”\marginpar{13} принимаемся мы превращать вечно сущие жизни; истребляем и рвём их вокруг.

Вместо слияния с миром — господствует: пожирание мира и раздробление мира; то есть: введение мира в желудок для накопления… жировых отложений. Человек XX века — безмясый скелет, опухающий жиром; вместо \emph{знания} и \emph{сердечного} отношения к жизни у него господствует два усвоения жизни: при помощи мозга и при помощи функций желудка; первое усвоение — \emph{„крап на ничто”} (т. е. крап электронов над бездной); и при помощи этого \emph{„дифференциального крапа”} слагает он на бумаге чертёжики пушек; усвоение же второе — чревоугодие; лишь оно одно доминирует в нём; малокровная мысль, превращённая в крап электронов, становится техникой чрева, изготовляя ему искусственные, многозубые челюсти крепостей, изборождённые пушками.

\section*{4.}

Моё окошко — в долину; цветущие, белокудрые вишни весною глядят из него; вечерами восходят закаты; в него свистит ветер — всю осень, всю зиму; над вершинами низкорослых деревьев — отчетливая черепица домов; дальше — дали, бегущие в линии голубоватых холмов; в\marginpar{14} голубоватом тумане — граница; будто бы иногда распахнётся там воздух: перед ненастьем особенно; и прорежутся темные гребни Эльзаса.

Вот оттуда то и летит:

— „Ру-ру-рууу”…

Порой отзываются стёкла окон; вдруг не выдержат; и — расплачутся; звук немецкой пушки я знаю: отчетливый, надоедливый звук; а вот это невнятное „у-у-у” — вероятно, французская пушка; говорят: из Мюльгаузена и из Бэльфора пушек доносится внятно до Дорнаха.

Так говорят эти пушки — дни, месяцы: девятнадцатый месяц; здесь, в Швейцарии, пушки молчат; но молчание здесь чревато глухим, наростающим взрывом; будут взрывы повсюду; и из груди, как жерла, оторвавшись от жил, точно бомба, взорвётся кровавое, обнажённое сердце;  человек в эти дни, точно пушка: заряжен он кризисом.

Тема кризиса сплетена с возрождением. Тема гибели мира связуема с темой рождения. Не случайны поэтому голоса, нас зовущие к выси духовной: переродиться пора!

Голоса Мережковского, Ибсена, Штирнера, Ницше, Владимира Соловьёва звучали. Звучит голос Штейнера. Выявляя нам нервы культуры,\marginpar{15} гласят очень внятно они о падении великолепных обломков культуры; и — о паденье домов: домов старого строя.

Дома — под обстрелом.

И под обстрелом, быть может, вся эта тишайшая местность: в первый месяц войны, Боже мой, что тут было; появились французы в предместьи Базеля, St. Louis; понадвинулись с севера немцы; и собирались вдавить из Эльзаса в Швейцарию, к нам, передовые французские части; поразвесили объявления о возможности битвы под Базелем; ожидали мы с часу на час здесь сигнала тревоги; по первому знаку сигнала должны были мы налегке пробираться — туда, через горы: черз кряжистый Гемпен, висящий над Дорнахом. По дорогам задвигались швейцарские пехотинцы; трещал здесь и там барабан; батареи уставились по направленью к границе; в пыли забелели султаны; и — фыркали лошади; заскрипели телеги с фуражем; а сумасшедшие, исступлённые кучки кричали, что надо бежать: нейтралитет будет попран. Говорилось тогда об обстреле домов; этот дом — не опасен, а этот — опасно поставлен.

Но опасно поставлен не дом, не окрестность, не даже кантон, не страна; вся культура — опасно поставлена; вся под обстрелом она. Все\marginpar{16} кумиры культуры — в опасности; изображения Вотана, Доннера, Логе — падут; гибель старых божеств волим мы. Старый бог, бог войны (alter Gott): должен пасть!

Рушатся представления о данной действительности; рушатся переживания её; пропадает в нас строй ощущений, будто \emph{„я”} в ней находится; пропадают реальные ощущения \emph{„я”}; действительность убежала от \emph{„я”}; утекла от него; как свинцовая гиря, стремительно погружается в глубину подсознания \emph{„я”}; его целостность точится всекипящим движением мира:

\begin{Verbatim}
В какие-то кипящие колёса
Душа моя, расплавясь, протекла.
\end{Verbatim}

\section*{5.}

Должное, реальное знание — в усвоении предмета узнания; а современное знание, сосредоточившись на методе, предмет упраздняет; предметом узнанья становиться метод; и вне метода — хаось: вращение газов в желудке.

Самые органы чувств порасшатаны современною культурою; и пропитаны — алкоголем, пропитаны — никотином; восприятия органов чувств — \emph{никотинны}; в них луг пахнет дымом; в восприятиях наших природа убита давно; пошлый\marginpar{17} рёв паровоза — неотъемлемая принадлежность обычного европейского пейзажа; и — линия телеграфных столбов; а фабричная гарь — принадлежность зари; естественных восприятий в нас нет; и оттого-то нам нужны абстракции доказательств и самая материя потребления; осязание, грубейшее чувство, оно только живо в нас.

Так предметы узнания уничтожаются нами: в процессе узнания распыляются в голове электронным пунктиром и раздробляются на зубах при введении в „чрево”. Вместо конкретного мира поэтому выростает мир в нас танцующих математических знаков (дифференциалов и функций), роящихся, точно грязные мухи над миром… желудочных отбросов.

Вот — подлинный, неприкрашенный образ материальной культуры; и вот результат: потребления мира природы, её раскромсанья на части; то есть разложение её — нами, в нас, вокруг нас; мир отсутствует в нас и вне нас; мы из мира повыпали; полетели над бездной в… действительность, несоизмеримую с некогда данной от Бога.

Действительность, нам грозящая, прорезается явственно, под покровами умерщвлённой природы; вот она показалась уже, пока видная нам в аппаратах, приборах и лупах, как мир… инфу\marginpar{18}зорий; но аппараты, приборы и лупы воистину суть: наши новые органы чувств; мы испортили наши природные органы; кто-то нам подарил мир искусственных органов, прилипающий к глазам и ушам, приростающий к безорганно висящему малокровному мозгу, протягивающему во все стороны, точно спрут, свои вялые разветвления нервов: высасывать соки природы; на обнажённые нервы, лишенные кожных покровов, насели нечистые мухи, роящиеся над сознанием нашим — математическим знаком; обнажённые нервы естественно бронируем мы: сталью, железом; бронированный сталью, бесформенный, нервный, безжалостный спрут — вот искусственный человек, приготовивший нам мировую войну.

\section*{6.}

Мир бактерий в микроскопах: „Вот я выйду из труб микроскопов; и расселюсь среди вас: бактерии заживут человеками; через трубу микроскопа вы свалитесь все в микроскоп; и — заживёте бактерией”.

Воистину, на земле мы — „как будто”; где её былой лик? Где её конкретная правда? Материальная культура — не культура земли: земля — идеально, конкретна, природна, естественна.\marginpar{19}

\section*{7.}

Люблю землю я: она — горная, кристаллически чистая масса; лишь её поверхностный слой — унавоженный перегной; унавоженный, дурно пахнущий перегной, перепачканный всевозможными отбросами, в представлении большинства производителей материальной культуры — земля. Но земля есть огонь: огонь лавовых струй; и с навозом не смешана.

И земля, — это горы.

Вспоминаю скитанья в горах; мыслишь там — ни о чём; ни о чём — свисты ветра.

Но \emph{никчёмные} мысли летают огромными ритмами; мыслью рушатся горы: в душе водишь думы; идешь себе; уж не смотришь: в полузакрытых глазах метаморфозы обставшего пейзажа сотворяются заново: полуобразом, полумыслью; там линия пиков змеится орнаментом мысли, овеянной ветром; ты — ветряный: в голове твоей ветер; останови его; и — фиксируй: он тотчас же уплотняется силлогизмами; сознанием проницаешь ты ритм вне-сознательной мысли.

Где-нибудь перекусишь; и — далее.

Сознание наблюдает, описывает проростание мыслей из красочных пятен фантазии; и — проростание в эти пятна тебя обстающих громад;\marginpar{20} громады поят тебя мыслью; она — чистая, кристальная мысль, там осевшая горной породой; и здесь вставшая — философемою, как вот эта долина; ты прошёл шесть долин; шестерично переменились рельефы; шесть систем философии пробежало вершинами: Гёте, Гердер, Новалис, Шлегель, Шеллинг и Гегель — прошли пред тобою.

Ты видел во-очию их. Твоя мысль ни о чём, пробуждённая в душу павшими пиками, осозналась; и — вот она: \emph{мысль и природа — одно}.

Человечна и мягко гуманна природа; в ней нет извращений; но человеческий взор озирает её плотоядно; человек современности на неё воззрился, как кошка на птичку; птичье пение — есть; но в зажаренном птичьем мясе нет пения, в „потреблённой” природе идея убита; и — мертвой материей противостоит она нам; материализм — вне природен.

Основа природы — природа идеи; и \emph{философия тождества} Шеллинга проверяема горным ландшафтом; слушай пенье потока; записывай точно, что встанет из пенья потока в душе у тебя; правда Гётевой мысли откроется явственно: \emph{„манифестация тайных знаков”} природы в умении понимать жизнь сознания; метаморфозы идей уподобляемы метаморфозе растительных организмов; метаморфоза идеи — кон\marginpar{21}кретна, точна, наблюдаема, описуема; и описание точной фантазии мысли и есть философия.

Вот гуманнейший лейт-мотив в мировоззрениях Гёте, Гегеля, Шеллинга, Гердера; и — других; и — вещая фантазия мысли (сознание и природа — единство в природе идеи) — гуманна; корни русского самосознания в ней, в этой мысли.

Бросать камнями в эти мысли, не значит ли — откровенно идти на разрыв с нас зовущей природою, эта природа — природа сознания нашего; но подчас преставления о природе в нас подменяются представлением о „трамвайной”, о „материальной” культуре, в которой природа — машина.

\section*{8.}

Всякий знает из нас — вот такую невнятицу: вдруг покажется, что в напряжениях материальной культуры на-двое разрывается жизнь, что машины бьют в пульсы слабеющей жизни железными пульсами; схватит такая минута стремительно в суете городов; и покажется вдруг, что вот он, фешенебельный господин Манекен с угла уличной вывески рекламирует немо толпе производство Гомункула; и — рост механической жизни покажется грозным наростом; пухнет опухоль городов — пора ампутировать опухоль: материальное тело жизни — раздуто чрезмерно.\marginpar{22}

Безотчетности эти переживаем мы все; и горизонты сознания возникают пред нами; горизонты гремят своим кризисом; материальную опухоль жизни пора ампутировать: человечеству угрожает гангрена.

Помню день.

Всё червонилось, багрянело; всё — рдело птичий свист — уносил; говоры оголтелых утёсов гремели, дрожали; рыдали потоки; землевороты бежали под небо и ясногранною вереницей плотнели в тенях, гребни резали небо; серебрились осколки их; прорезались покровы природы, прорезались навстречу природе — природа сознания; и природа в природу сознания поглядела, как Сфинкс; земли не были землями: одухотворились и жили они, как природа идеи.

\section*{9.}

Материальная культура давно отошла от земли потому, что земля — идеальная; наш мир — мир искусственных аппаратов, понятий, стремлений и похотей; этот мир — не земля, а пожалуй — какая-то „X” планета, быть может созвездие Пса, не созвездия Солнца… Не дети мы Солнца.

Солнце прежнего мира (откуда мы выпали) разве что бьётся в нас; мир идеально-конкретный, т.е., небесно-земной — разве что в нашей\marginpar{23} совести, в беспредметно поющей, ритмической Аполлоновой музыке наших душ; может быть — в поэтической сказке; подлинная земля — только он; в нём — остаток конкретности.

Мы его гнали.

Мы подзывали иную действительность; мы столетия сотворяли её; мы плели себе горькую участь: заплетали нити судьбы; мы расплавили землю в \emph{„не сущие”}, электроны и атомы; диссоциация мира в картине научных понятий имеет условный, технический, вспомогательный смысл; мы осмыслили тот смысл: возвели его в перлы „идейного” творчества; перед нами восстала картина — разрыва действительности; распадается, разрывается человек — под говором пушек.

\section*{10.}

Первое впечатленье от Сфинкса:

— „Старая обезьяна, урод, эфиоп”.

Последнее впечатление:

— „Ангел”.

У подножия Сфинкса бывал очень часто я;\marginpar{24} тайна слиянности в нём Апполона и эфиопа меня поражала всегда; эта тайна вырвала его из веков; этой тайною он притянут к XX веку до нашей эры; и — после; так оба XX века пересекаются в Сфинксе: а прибои культур пеною разбиваются о него; он глядит — из под греческой маски; и из под рожи современного футуриста-художника; злободневность в нём разорвана в Вечность; и в нём Вечность сама — злоба нашего дня; в нём слиты полярности пола (он — δΣφΙγξ; он — ή). Нет ничего злободневнее кучи трухлявого камня, на которой изваяна эта старая голова.

Ряд статей одного писателя (к сожалению разменявшего свой талант) вспоминался мне в Египте: произведения Достоевского в нём сопоставлены с… египетской графикой, как плоды одинаковых переживаний и бурь; если так это, то интимное видиние Египта мы носим с собою; его тайны — на улице, где-нибудь между Литейным и Невским, а не… в булакском музее; тысячелетия прошлого не вне нас, в — в нас они с нами; и в движении нашего пальца, в улыбке, в манерах, в цилиндре, во всём строе жизни — осуществившийся синтез Египта, Халдеи, Ассирии и т.д.; откровение ассирийского духа не в кропотливейшем изучении мало понятных письмен, а в — газетной статье, может быть; эсотерика — в\marginpar{25} \emph{умении видеть}: на улице — улицу; в храме — храм; под злободневною рябью — океанические глубины все те же, что и под спудом мистерий; пиджаки таинственных мантий; и несомненно мне, что та же священная тайна построила формы фараоновой шапки и каски пожарного; в линии орнаментального завитка на дешёвеньком \emph{ситчике} — линии священнейших таинственных знаков; в бегстве от \emph{„улицы”} правды нет: в умении видеть на улице \emph{сфинксовы тайны} — посвящение в мудрость XX века. Собственным ликом на вас смотрит Рамзес II, фараон — из под стекла, в музее среди каирских кварталов, напоминающих мне Париж, а его двойник, полисмен, феллах стилизованным \emph{египетским} жестом поднимает на улице белую полисменскую палочку среди автомобильного тока; подлинный лик Рамзеса — феллахский; между Рамзесом и статуями Рамзеса в Мемфисе — ни малейшего сходства нет: впечатления эпохи Рамзеса посещают вас и на улице; вы бываете странно выбиты — изо всех культур и историй.

\section*{11.}

Кто не знает этого переживания во время горных подъёмов? Вы живёте в маленьком городке; вы охвачены его жизнью; и вы в неё впаяны; вы\marginpar{26} сидите в кафе; и со всеми вместе склоняетесь вы над газетою; переживаете события мировой войны, бродите по бесконечным уличкам, над которыми приподнялся далёкий, всё тот же, пейзаж — точно фон декорации; и всё то же озеро плещет, бросая о берег все те же лимонные корки.

Вот теперь вы уходите в горы.

Смотри же:

— Озеро опустилось под ноги и медленно сжалось; разгладилась его рябь, будто скомканный лист оловянной бумаги отполировала детская ручка; и проступили: глубинные, невыразимые тоны вод; так проясняются глаза человека в минуту задумчивости и изливают лазури, из извечного устремляясь в века; тает так в современности злоба дня; из под пены её наблюдаете вы: в современности тысячелетия прошлого.

Изменяется всё: цвет воды, дома, люди, рельефы; бесконечности каменных домовых квадратов и кубов теперь сжаты глубоко под вами — на чётком мысочке; а, казалось, дальние горы — повытянули свои главы, раздались громадно плечами и перегнулись над озером — прямо к вам: на вас смотрят в упор; то, \emph{чем смотрят} они — не война, не события городка, города, столицы, страны, континента, эпохи, периода времени;\marginpar{27} о современности, понятой в нашем смысле, не может быть и речи; и — тем не менее: \emph{этот взгляд} современен; и эти пространства утёсов злободневно кипят множеством неизливных ручьёв; Ассирии, Вавилоны, Египты кипели своей злободневностью; и — откипали бесследно; а эти летящие струи кипели всё так же; и — \emph{тем же} кипели; кипение этой жизни — не мертвенно; мертвеннее — сиденье в кафе; даже мертвеннее — война; всё текущее остановится в XXV столетии, перенесётся в музеи (если будут музеи); на мыску там, под вами, будут выситься, может быть, пятиугольные здания со странными куполами; а кипенье потоков, взгляд горных громадин — останется тем же всё; \emph{тоже} — вызовет он в душе, что — в этот миг происходит; переживание Ганнибала, может быть, стоявшего здесь, вы узнали теперь — с математической точностью; человек XXV века, вы, Ганнибал и пещерный доисторический человек, пересеклись теперь в одном пункте души; и то, в чём вы все пересеклись, есть вечное; кристаллизация культур, эпох, современностей, и довлеющих дневных злоб выкристаллизовались из подобного мига; вы теперь — не над маленьким швейцарским местечком, а над всеми культурами: бывшими, грядущими, сущими; вы у себя самого; потому что\marginpar{28} вы — в космосе: и космическая картина сознания из под порога сознания городского — она перед вами; в дневнике происшествий перед вами разъялися смыслы; загласили они громовыми воплями серафимов; обратно: с глубинного переживания Заратустры слетает покров: и — покрывало Изиды откинуто. Вы теперь — мощный горн, где потенции всех небывших и бывших культур протекают в расплавленном состоянии; и — погибни история: из души человека, как из вулкана, поднявшися, вытекут все культуры с тайнами и бестайностями, все кристаллизации истин, всё несчётное число их покровов; тут вступаете вы в ту область, где разумение законов и истин не в них, а в самом законе законов, их строющем: в ритме строения, в законе метаморфоз переживаний, мыслей, мод, стилей, линий; здесь сознание перелетает порог, потому что и нет его у сознания; порог сознания сознанию не присущ; порог сознания есть всегда лишь граница, извне застилающая мои кругозоры: не граница зрения, а предмет пред глазами стоящий, как вот эта стена моей комнаты, улицы, застилающей зренье горы; зрение моё видит звёзды; в зрении миллионно-вёрстные дали перемогаю свободно я; в сознании — тоже; надо только выйти из рамок; уметь развивать мускулы, спо\marginpar{29}собные приподымать меня за порог передо мною положенных стен, и мне не присущих.

\section*{12.}

Многие жители городов не покидают города вовсе; а для многих больных представление о пространстве связано с представлением о четырёх жёлтных стенках, вытарчивающих у них за окошком.

По отношения к сознанию — то же; мы все больны параличами воли сознания; оттого-то мы ему искусственно полагаем границу — в пространстве и времени; эта граница в пространстве — стена; и эта граница во времени — злободневность; злободневность и процинциализм нашей жизни из поколения в поколение отпечатали в нас свои представления о границе сознания и о пороге сознания; границ сознанию нет; а пороги сознания побеждаемы — в пространстве и времени.

Ведомы нам дали здёзд; и — непосредственно ведомы; а непосредственное переживание тайн истекших культур — звёзды Индии, Персии, Египта, Халдеи — будто бы нам неведомы; и будто бы: чтоб понять нерв истории надо кануть в пыль музейных архивов; но музейные данные не Египту научат нас; а египетской пыли; сам Египет — он в нас; кровь от крови его, плоть от плоти\marginpar{30} его — мы: следует только твёрдо отставить систему фальшивых порогов, в которых будто бы сознание наше заключено, как в тюрьму; эти „пороги” — есть пыль злободневности: её ненужные отбросы; в своих нервах она — та же древность: которую будто бы она заслоняет, и то же грядущее, которое будто бы и вовсе неведомо нам; оно неведомо в своей „пыли”; и оно с нами — в сути: в ритме душевные глубины обнажены нам бывают порою; надо только суметь произвольно их открывать и описывать явления глубинной жизни сознания; в душе каждого обнажаем колодезь, у которого нет индивидуального дна и которые есть выход одновременно в небо духа и космосов; как по тем же космическим принципам образовались планеты всех солнечных систем всех вселенных — точно так же слагаются и слагались фазы истории; и закон фаз во мне; улови его — история встанет передо мной разоблачённой в её объясняющем стержне; а понимание многообразий мелодий её — только чтение нот; суть не в нотах, а в принципе нотного чтения; и умение чтения знаков событий, искусств, стилей, мод, философий, религий и душевных движений — в умении опускаться в себя, открывая горы и пропасти; кажущееся противоречие между жизнью в себе и в других устраняет первый\marginpar{31} опыт внимания к своим собственным душевным движениям; кажущаяся их хаотичность, невнятность, бесстроица; т.е. всё то, что любители звучных слов именуют \emph{„сублиминальным полем”} сознания — на сознании насевшая пыль: принимаясь за чистку пыли вы пыль поднимаете; но именно поднятие пыли, есть условие очищенья от пыли; при внимательном взгляде внутрь исчезает весь хаос; окружающее отражено вами и чище и чётче; в умении отражать — умение понимать; душа — зеркало мира; замутнено оно — мира нет.

\section*{13.}

Злободневность летит перед нами на своих острейших углах; быт и вкусы кричаще ломаются на протяжении трёх-четырёх поколений; так в малых масштабах линия исторической жизни рисует контрасты нам; а в огромных масштабах господствует закон сходств; многообразие изломов истории в её мыслях, вкусах, костюмах, архитектонике стилей, быта, привычек — образует лёгкую рябь по отношению к основной неизменяемой прямой линии времени; на известной стдии восхождения стирается эта рябь: и отражается в зеркале жизни душа человека; жест куафёра и жесты Фридриха Ницше рази\marginpar{32}тельно противоречат друг другу, пока вы в злободневности, верней в её отбросах; но погружаясь в себя или олько чаще карабкаясь в горы: вы начинаете понимать оба жеста, как модуляции единого душевного жеста; вся бесстроица мимик теперь — метаморфоза немногих нот мимики, ведомых вам изнутри и глядящих извне на вас: улыбкою египетских статуй, улыбкой Джиоконды и…, может быть, вами  виденной булочницы; понимаете вы: что душа древней критянки не так-то закрыта от вас, раз XVIII век, столь вам близкий, повторяет пред вами древнейшие критские моды, как гласят нам раскопки.

Возвращаяся с гор, только чутче уловите вы всё \emph{горное} в долинной деревне; возвращаяся из себя в злободневность, вы увидите — вечность в ней.

\section*{14.}

Ницше, модница, мировая война, дневники происшествий, раскопки на Крите, священные письмена и текущий судебный процесс — злобы дня Вечности, потому что \emph{вечное} — злободневно; умени переносить тысячелетья, в мгновенья, умение переживать мгновенное в вечном — злободневнейший нерв злобы дня, потому что Сфинкс истории стоит перед нами, Эдипами, и предлагает нам\marginpar{33} свои загадки и тайны: злободневными темами, и злободневнейший ответ злобы дня — есть наш ответ Сфинксу.

Наш ответ есть судьба.

\section*{15.}

Человеку пора призадуматься над судьбой человека; и — европеец задумался; человеку пора апеллировать к смыслу жизни, который в нас вписан. Как в бы труп убитого, конкретного мира не вошёл дух иного, \emph{недолжного} мира, противного нашей совести?

Нужно твёрдо нащупать в нас \emph{человечество}; например: не ангельство, не машинность; в настоящее время испуг пред машиною вызывает в иных обращение к религиозным истокам; но возвращение к Богу у многих носит машинный характер: многие защищаются Богом, противополагая его звериному лику природы, противополагая божественность самому человечеству; и обращение к Богу носит характер возвращения вспять.

Мы должны вернуться к истокам великого Гуманизма: возрождение, традиции Возрождения, его широкий, вполне гуманный размах — вот что нужно усвоить нам. Схоластическим прением с бездушной машиною и преданием всей культуры\marginpar{34} костру инквизиции мы ускорим лишь наш печальный конец; отдавая свободное, автономное лицо машине в исканиях нового „Ангельства”, мы приблизим машину к себе; возвращение к средневековым суевериям обернёт мир машин нам в бесовскую рать; возврат суеверий вернёт навождения; и аппарат — будет нам: Господин Аппарат; Господин Аппарат будет новым Hircus Nocturnus'ом на шабашах человечества. В образах и подобиях механической жизни напечатлелые средних веков; механическая культура тесно связана с тёмными сторонами схоластики; схоластика возродилась в \emph{методике}, в \emph{методологии логики}; но она пустила более глубокие корни. К схоластике проведенья понятия по методической шкале присоединилась схоластика проведения нашей жизни по машинному ряду.

Средневековые вкусы новейшего модерниста — естественный дополнительный цвет к… механическому мировоззрению современности; от Рожера Бэкона — к Лойоле; и от Лойолы — к фабричной машине: вот путь нашей жизни. Соединение исповедальни Лойолы с фабричной конторой — сюрприз, ожидающий нас.

Кризисы материальной жизни — колебание её идейно-конкретной подпочвы; идеология — резуль\marginpar{35}тат конкретного творчества; идеология материальной культуры, разлитая в мире, — подлинная причина войны; а война, — выражение внутренно скрытой болезни, глодавшей вселенную: нечто вроде лихорадочной сыпи, проступившей из крови — на коже; тут втиранием мази ничем не поможешь; изменений крови — вот корень лечения; он — в перемене ритма пульсаций.

Наше сердце пульсирует метрами многогромных колёс и метрами дребезжащих трамваев. Пусть же оно запульсирует метрами песен, пусть он осаждается бытом: анемия сознания перестанет изламывать жизнь; механика пресуществится в органику; в противлении человеческой совести механизму — трагедия сознавания; и в ней — кризим; внешние выраженья его — междуусобия, войны, болезни, убийства; и, наконец, — помешательство.

Механическое прекращение нас обставшего кризиса будет безплодным втиранием мазей — лечением прокажённой кожи культуры вместо лечения заболевшего сердца.

Грохоты горизонтов сознания не прекратятся: они — это — мы; мировая война — alter ego.

\section*{16.}

Вот сейчас я пишу, а „гром” — непрерывен: глухой говор поднимется; глухой говор потом\marginpar{36} упадёт; а сегодня — многие говоры, перебегая друг в друга, сливаются в басовую, глухую и тяготящую дрожь:

— Уу-у-ууу…

Так гремело всю зиму, всю осень, всё лето, всю весну, всю прошлую зиму, всю прошлую осень и валились прохожие между грудами черепитчатых домиков; и валились из окон, из груды перин, и — уставляясь в закат… с выражением, точно из них выпирало сплошное, тупое, больное, „оно” — огромное, неживое какое-то.

\begin{Verbatim}
Ты подвиг свой свершила прежде тела —
Безумная душа…
Под веяньем возвратных сновидений
Ты дремлешь; а \emph{оно}
\emph{Бессмысленно глядит}, как утро встанет,
Без нужды ночь сменя;
Как в мрак ночной бесследно вечер канет, —
Венец пустого дня… —
\end{Verbatim}

\hspace{0.45\textwidth}— говорит Барытынский, — и вот \emph{„оно”}, это тело, говором пушек утомлённой души глядит \emph{„оно”}… в листики местной газеты, оповещающей мир и постановке нового фонаря между Дорнахом и Арлесгеймом; и — о битвах в Европе.

Глядя на жизнь, распростёртую передо мной в тихих сёлах, мне кажется, что действитель\marginpar{37}ность рухнула, как дебелое тело швейцарца, в перины: рассыпалась в атомах; пляска атомов сонного тела не жизнь; это — казнь; это — месть за изъятие мира из мысли; мир, изъятый из мысли, — ни мир и ни мысль; он — не он, не она, а какое-то неживое \emph{„оно”}, безучастно вперенное в нас и свершившее подвиг свой; такой взгляд — тихий взгляд помешательства.

Не странно ли: до-военная суматоха напоминала сплошной „тихий час”; обыкновенные грохоты жизни как будто теряли свой голос: и настала она — тишина; иль вернее — \emph{„оно”}, — то „оно”, что уже девятнадцатый месяц томит меня здесь в Базельланде.

\section*{17.}

Тишина русских ширей — прозрачная, ясная, внятная, окрылённая в грусти своей: тишина ожидания.

В моём тихом углу бесстремительно всё; каменно тишина припадает: поникаете вы у себя на дому; поникаете в ожесточённо рыдающем ветре; поникаете в летнем безветрии; и все — точно валятся на дорогах меж грудами черепитчатых домиков, друг другу подобных; валятся и молчат: тяжело говорят, тяжело глядят исподлобья; и — тяжело поступают.\marginpar{38}

Тишина Базельланда — \emph{„оно”}; про \emph{„оно”} поют петухи; и отбивает отчётливый колокол — всё про то, про одно: про \emph{„оно”}; и \emph{„оно”} здесь во всех; и \emph{„оно”} здесь во всём; и все — в „нём”.

По ночам „оно” почивает в перинах, не слушая, как тихо дзынкнет окошко, когда… громыхнёт с горизонта.

Отяжелевший, бессмысленный, неосмысленный мир!

— И тяжёлая мысль, потерявшая смысл!

\section*{18.}

Ни одного события!

Всё — мертво; всё — спокойно; издали лишь военный оркестр вечерами наигрывает грациозные вечерние зори: прекратилась стремительность суетливых движений военных отрядов; изредка, загорелая кучка солдат пробушует на улицах деревушки; и — пропоётся тут песня; разговоры о том, что, мол, выстрел — убитые жизни, — разговоров таких теперь нет (попривыкли мы к „говору” пушек); говорится — о керосине, об угле, о сахаре: подорожали продукты; подорожают ещё: то ли будет!

— „Ру-рууу” громыхнёт с горизонта.

Тянутся сонливые дни; и проходят бессонные ночи; никогда ещё меч сомнения не рассекал\marginpar{39} наши души такою огромною болью; никогда ещё разделение не секло нас этой секущею силою; кризис сознания пересекается с кризисом жизни; были же голоса — отчего их не слушали?

Я в туманную ночь открываю окно; из туманной ночи вылетают снопы: световое пятно загорится на тучах; разбегается, угомониться не хочет: сигналы.

Затворяю окно: продолжаю писать.

\section*{19.}

Связь вещей — в моём «Я»; эта связь есть сознание; «знания» — члены связи — неизменяемы: изменяема комбинация их; она — в ритме; ритм знаний — сознание; в модуляциях протекает оно, оплодотворяя нам знания; в оплодотворениях творчески раскрывается наше «Я»; все понятия знаний рассудочны; вне их вяжущей связи они рубят действительность на бессвязно-текущие части; в переложениях и сочетаниях, во взаимном их контрапункте — в сознании — окрыляется разум; в сознании — цельность «Я»; оно — знание собственно; оно — родит свои формы; в оформлении, в \emph{«ставшем»} нет жизни; в \emph{становлении} строится \emph{ставшее}; в ритме строится форма; в сознании — знания; сознание — центр текущего организма вселенной; вне его организм этот — труп.\marginpar{40}

Из зерна бежит стебель; из стебля наливается колос, а в колосе — зёрна; и — так далее, далее; \emph{прежде} и \emph{после} — отсутствуют; отношение сознания к миру и мира к сознанию — отношение колоса и зерна; из раздельность — абстракция; мысли мира и мыслимый мир суть единство в сознании; задрожи оно — рухнет мир.

С нами кризис сознания; и — стало быть: кризис мира; земля — в знамениях; стало быть: будут знамения в небе; твердь земли и небес — она дрогнула в нас; сознание наше мутится: диссоциация материи налицо; о ней кричит физика; и — диссоциацию духа являют нам наши вкусы — в искусстве и в жизни; футуризмы, кубизмы нам убивают искусства; вырождение, кретинизм убивают нам нашу жизнь.

И пора нам спросить со всею ответственной строгостью: что есть знание? Чем должно оно быть? И — каково оно в \emph{«знаниях»}?

\section*{20.}

Знание — брак «Я» и мира; \emph{«и познал жену»}… говорится нам в Библии; знание — в слиянии с узнаваемым.

Знание цветка обнимает знание его функцией, тычинок, и имя, произнесённое на звучной латыни; кроме того: предполагает умение пережи\marginpar{41}вать себя в нём; полевую лилию знать — значит стать: полевою лилией в поле; видеть солнце, как лилия; узнать лилию в спиртовом препарате, это значит — придти к убеждению, что она — пахнет спиртом.

Всякое абстрактное знание нас ведёт к суррогату: к лилии… на бумаге, к рисунку и к схеме; такой лилии — нет; её надо доказывать.

Доказательства Божьего бытия — суррогат жизни в Боге; умер Бог в сердце нашем; и мы начинаем — вести разговоры о Боге.

Пусть на полюсе пальма — мечта: она \emph{есть} в жарких странах; полюс ведает лёд; для него всё иное — фантазии; но вера в фантазию, в пальму, покоится на возможности наблюдения пальмы в Египте; отношения \emph{знаний и вер} — отношения египтян к эскимосам; на экваторе снег мифичен; грозовая туча — миф полюcа; предмет \emph{знаний и вер} переменчив; география сознания нашего — она неизменна.

И абстрактна граница меж верой и знанием; невозможны без опыта ни та, ни другое; вера в знании — есть; и есть знание веры: дела веры в опытах; мертва вера без дел, мертво знание без опыта.

Принцип нужен для опыта; вне его он —\marginpar{42} абстрактен; знание без веры абстрактно; и вера без знания — сказка.

Рассудочно рассечение жизни абстракцией принципа; действительность раскрошена ею в формах; материализм есть абстрактное \emph{крошево} трупа мира на мельчайшие части; спиритуализм вылагает принцип из жизни: и уплотняет его — в неподвижностях догмата; и — \emph{каркас} духа — догмат — без плоти и духа: таковой \emph{каркас} — пуст; догматический спиритуализм — вытравление плода жизни из жизни; материализм — убивание самой жизни; разделённые вера и знание, нам грозят оскоплением и разложением жизни; разделённые, вера и знание нам гласят, что \emph{духовное знание} невозможно: непостижен де дух, бездуховно де знание; и, молясь, неизвестности, мы живём в мертвечине; остаются нам знания неверны; остаются нам веры незнаемы.

Мы хотим умных вер, волим верное знание.

\end{document}
